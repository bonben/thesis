%!TEX root = ../my_thesis.tex
\chapter{Caratérisation et identification des erreurs résiduelles} % (fold)

Texte intro

\vspace*{\fill}
\minitocTITI
\vspace*{\fill}
\newpage

\section{Existence des erreurs résiduelles}
Comme présenté dans le chapitre premier, l'apparition du plancher d'erreur dépend de la distribution des mots de codes 
possédant un faible poids \cite{distance_spectrum}. Il a été déjà remarqué dans \cite{takeshitaBCH} que les trames 
erronées 

\begin{equation}
P(m) \approx \frac{\displaystyle\sum\limits_{d, \frac{W_d}{A_d}=m} A_d\cdot \exp\left(-d R \frac{E_b}{N_0}\right)}
                  {\displaystyle\sum\limits_{d} A_d\cdot \exp\left(-d R \frac{E_b}{N_0}\right)}
\label{eq:be}
\end{equation}