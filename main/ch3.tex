%!TEX root = ../my_thesis.tex
\chapter{Caractérisation et identification des erreurs résiduelles} % (fold)

Texte intro

\vspace*{\fill}
\minitocTITI
\vspace*{\fill}
\newpage

\section{Existence des erreurs résiduelles}
Comme présenté dans le chapitre premier, l'apparition du plancher d'erreur dépend de la distribution des mots de codes 
possédant un faible poids \cite{distance_spectrum}. Plus encore, il a été montré que la zone du plancher d'erreur émanait 
de la présence \emph{d'erreurs résiduelles} \cite{takeshitaBCH}. 
La Figure \ref{fig:befe} présente l'évolution pour différentes valeurs de SNR du nombre de bits erronés par trame 
erronée pour quatre turbo codes du standard LTE et un du standard CCSDS. Dans tous les cas l'algorithme EML-MAP itérant 
jusque 8 fois est considéré. Il est visible que dans la zone de convergence, de nombreuses erreurs à l'issu du décodage
sont présentes. En revanche, dès lors que le décodeur fonctionne dans la zone du plancher d'erreurs, dans tous les cas
considérés, le nombre d'erreurs binaires par trame est inférieur à 10 en moyenne. De là, vient leur nom d'erreurs résiduelles.

\begin{figure}[!b]
	\centering
	\includegraphics[width=.8\textwidth]{main/ch3_fig/be/tikz/befe.pdf}
	\caption{Évolution du nombre moyens d'erreurs binaires par trame erronée pour différentes valeurs de SNR et différents
	turbo codes. Algorithme EML-MAP itérant jusque 8 fois. \label{fig:befe}}
\end{figure}

Une caractérisation de la distribution de ces erreurs 
résiduelles a été proposée dans \cite{residual_errors}. Cette dernière est basée sur les fonctions recenseuses de poids
(Weight Enumerator Functions ou WEF en anglais). En posant $P^{\langle ML\rangle}(m)$ la probabilité que $m$ bits soient 
erronés dans une trame, sachant que la trame est erronée après décodage à maximum de vraisemblance, il vient :
\[P^{\langle ML\rangle}(m) \simeq \frac{W^m~A_m^{\langle CP\rangle}(Z)\vert_{W=Z=e^{-RE_b/N_0}}}{\sum\limits_{w=1}^N W^w~A_w^{\langle CP\rangle}(Z) \vert_{W=Z=e^{-RE_b/N_0}}}\]
où $A_w^{\langle CP\rangle}(Z)$ est le WEF conditionnel du turbo code considéré \cite{benedetto_unveiling}.

De façon empirique (cf. Annexe \ref{sec:annWEF}), il a été montré qu'en considérant que les séquences d'information 
générant un mot de code de poids $w$ ont majoritairement le même poids, l'équation précédente peut alors être transformée en : 
\begin{equation}
P^{\langle ML\rangle}(m) \approx \frac{\displaystyle\sum\limits_{w, \frac{W_w}{A_w}=m} A_w\cdot \exp\left(-w R \frac{E_b}{N_0}\right)}
                  {\displaystyle\sum\limits_{w} A_w\cdot \exp\left(-w R \frac{E_b}{N_0}\right)}
\label{eq:be}
\end{equation}
avec, comme déjà présenté dans le Chapitre Premier, $A_w$ le nombre de mots de codes de poids de Hamming $w$. La 
multiplicité des bits d'information $W_w$ est la somme des poids de Hamming des $A_w$ séquences binaires générant des
mots de codes de poids de Hamming $w$. 

Ainsi, le spectre de distance d'un turbo code permet de calculer la probabilité 
d'obtenir $m$ erreurs binaires dans un trame erroné en considérant un décodage ML pour un taux d'erreur suffisamment 
faible et un canal AWGN. De part la présence du terme en exponentielle, plus $w$ est distant de $d_{min}$, moins la 
multiplicité associée a un impact sur la distribution des erreurs résiduelles. Néanmoins, ceci peut être contre balancé 
par une multiplicité très importante (de l'ordre du millier).

Dans la section suivante, grâce à l'obtention des spectres de distances de différents turbo codes standardisés, une 
comparaison est menée quant à la distribution du nombre d'erreurs dans les trames erronées dans la zone du plancher d'erreurs.


\subsection{Distribution des erreurs résiduelles pour des turbo codes standardisés}
L'existence des erreurs résiduelles dans la zone du plancher d'erreur vient d'être établie. Une proposition d'obtention 
de la distribution de ces erreurs a été conjointement proposée. La Figure \ref{fig:be} présente les taux d'erreurs binaires pour 
différents turbo codes des standards LTE et CCSDS.

\begin{figure}[!ht]
	\centering
	\includegraphics[width=\textwidth]{main/ch3_fig/be/tikz/be.pdf}
	\caption{ \label{fig:be}}
\end{figure}


