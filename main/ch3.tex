%!TEX root = ../my_thesis.tex
\chapter{Des oscillations du décodage itératif} % (fold)
\label{cha:chapter_3}

Dans ce chapitre, on présente le SC et le décodage  à la Benjamin

\vspace*{\fill}
\minitocTITI
\vspace*{\fill}

\section{État de l'art} % (fold)
\label{sec:_tat_de_l_art}

\subsection{Convergence} 
\label{sub:convergence}


\subsection{Critère d'arrêt}
\label{sub:crit_re_d_arr_t}



\section{Mesures} % (fold)
\label{sec:mesures}

\subsection{Polarisation des APP}
\label{sub:polarisation_des_app}

\subsection{Nombre d'erreurs en fonction des oscillations}
\label{sub:nombre_d_erreurs_en_fonction_des_oscillations}


\subsubsection{Extrinsèques} 
\label{ssub:extrins_ques}

\subsection{Conclusion} % (fold)
\label{sub:conclusion}

\section{Le Self-Corrected} % (fold)
\label{sec:le_self_corrected}

\subsection{Approche originelle} % (fold)
\label{sub:approche_originelle}

\subsubsection{Pour les LDPC} % (fold)
\label{ssub:pour_les_ldpc}

\subsubsection{Transcription aux turbo codes} % (fold)
\label{ssub:transcription_aux_turbo_codes}

\subsection{Études sur différents turbo codes} % (fold)
\label{sub:_tudes_sur_diff_rents_turbo_codes}
avec random interleaver

\subsection{Conclusion} % (fold)
\label{sub:conclusion}

\section{Oscillations pour le redécodage} % (fold)
\label{sec:Oscillations pour le redécodage}

\subsection{État de l'art} % (fold)
\label{sub:_tat_de_l_art}

\subsubsection{Correction Impulse Method} % (fold)
\label{ssub:correction_impulse_method}

\subsubsection{Propostion Benjamin LDPC} % (fold)
\label{ssub:propostion_benjamin_ldpc}

\subsection{Expérimentations} % (fold)
\label{sub:exp_rimentations}

\subsection{Conclusion} % (fold)
\label{sub:conclusion}


