\documentclass{standalone}

\usepackage{amsmath}
\usepackage{pgfplots}

\pgfplotsset{compat=newest}
\usepgfplotslibrary{colorbrewer}
\usepgfplotslibrary{groupplots}
\usetikzlibrary{matrix, positioning, patterns}
\pgfplotsset{
/pgfplots/ybar legend/.style={
/pgfplots/legend image code/.code={%
\draw[##1,/tikz/.cd,bar width=0.1cm,yshift=-0.2em,bar shift=0.5*\pgfplotbarwidth]
plot coordinates {(0.5*\pgfplotbarwidth,0.6em) (2.5*\pgfplotbarwidth,0.4em) (4.5*\pgfplotbarwidth,0.2em)};},
}
}
\input{colors}

\begin{document}
\begin{tikzpicture} 

\begin{axis}[
  name=plot1,
  height=0.56\textwidth,  width=0.5\textwidth,
  axis y line*=left,
  ylabel={nombre de cycles},
  symbolic x coords={1024,512,256,128,64},
  xtick=data, xlabel={N},
  ybar,
  ymax=15000, ymin=0, enlarge x limits=0.15, bar width=0.3cm
  ] 
\addplot[Paired-1, fill, fill opacity=0.3, postaction={pattern color = black!80!Paired-1!70, pattern= north east lines}] 
  coordinates {(1024,14030) (512,7058)  (256,3893) (128,2457) (64,1349)}; \label{plot:A57_1_2}
\addplot[Paired-5, fill, fill opacity=0.3                                                                              ] 
  coordinates {(1024,6023)  (512,3260)  (256,1639) (128,865)  (64,342) };  \label{plot:ASIP_1_2}
\end{axis} 

\begin{axis}[
  height=0.56\textwidth,  width=0.5\textwidth,
  legend style={at={(1.2,-0.15)}, 
                anchor=north,
                legend columns=2
                }, 
  axis y line*=right,
  axis x line=none,
  ymin=0, ymax=5, enlarge x limits=0.15,
]
\coordinate (legend) at (axis description cs:0.45,0.97);

\addplot[smooth,mark=*,Paired-9]
  coordinates{
    (0     ,2.3)
    (0.0148,2.2)
    (0.0295,2.4)
    (0.0441,2.8)
    (0.059 ,3.9)
};
\end{axis}

\begin{axis}[
  name=plot2,
  at={($(plot1.east)+(1.6cm,0)$)},anchor=west,
  height=0.56\textwidth,  width=0.5\textwidth,
  symbolic x coords={1024,512,256,128,64},
  xtick=data, xlabel=N,
  ybar,
  axis y line*=none,
  ymax=15000, ymin=0, enlarge x limits=0.15, bar width=0.3cm
  ] 
\addplot[Paired-1, fill, fill opacity=0.3, postaction={pattern color = black!80!Paired-1!70, pattern= north east lines}] 
  coordinates {(1024,11210) (512,6245)  (256,3201) (128,1903) (64,1159)}; \label{plot:A57_1_4}
\addplot[Paired-5, fill, fill opacity=0.3                                                                              ] 
  coordinates {(1024,5200)  (512,2904)  (256,1523) (128,669)  (64,338)};  \label{plot:ASIP_1_4}
\end{axis} 

\begin{axis}[
  height=0.56\textwidth,  width=0.5\textwidth,
  at={($(plot1.east)+(1.6cm,0)$)},anchor=west,
  axis y line*=right,
  axis x line=none,
  ymin=0, ymax=5, enlarge x limits=0.15,
  ylabel=rapport du nombre de cycles
]
\addplot[smooth,mark=*,Paired-9]
  coordinates{
    (0     ,2.2)
    (0.0148,2.2)
    (0.0295,2.1)
    (0.0441,2.8)
    (0.059 ,3.4)
};\label{plot:reduction}
\end{axis}

  \matrix [
      draw,
      matrix of nodes,
      anchor=north,
      fill=white
  ] at (legend) {
      A57     & \ref{plot:A57_1_4}   \\
      ASIP    & \ref{plot:ASIP_1_4}  \\
      Ratio   & \ref{plot:reduction}    \\
  };

  \node [align=left, text width=5.4cm] at (2.5,-1.4) {(a): Rapport et nombre de cycles pour un code de rendement 1/2} ;
  \node [align=left, text width=5.4cm] at (8.5,-1.4) {(b): Rapport et nombre de cycles pour un code de rendement 1/2};

\end{tikzpicture}
\end{document}
