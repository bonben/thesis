%!TEX root = ../my_thesis.tex
\chapter*{Conclusions et perspectives}
\markboth{Conclusions et perspectives}{Conclusions et perspectives}
\addcontentsline{toc}{chapter}{Conclusions et perspectives}

\section*{Résumé et comparaison des différentes approches}

Les travaux présentés dans ce manuscrit portent sur l'implémentation d'algorithmes de décodage de codes polaires sur des architectures programmables.

Dans le deuxième chapitre, des implémentations logicielles des algorithmes de décodage de codes polaires à liste sont proposées. L'originalité de ces décodeurs logiciels tient à leur forte flexibilité. Cette flexibilité est inédite en comparaison des travaux préalables publiés dans la littérature. Tout d'abord, les données internes de l'algorithme sont représentées en virgule fixe, ce qui diminue l'empreinte mémoire et permet de réduire le temps nécessaire au décodage d'une trame. De plus, l'élagage de l'arbre de décodage est une simplification indispensable pour atteindre de hauts débits et de faibles latences de décodage. Cet élagage, dans les décodeurs proposés, est dynamiquement configurable. La flexibilité de l'élagage permet des compromis intéressants entre le débit, la latence et les performances de décodage. Grâce à différentes optimisations, la flexibilité et la généricité du décodeur sont atteintes sans sacrifier le débit ou la latence de décodage. Une des versions de l'algorithme à liste, l'algorithme FASCL, permet même de dépasser le débit des implémentations de l'état de l'art sur les architectures x86. Des résultats d'implémentation logicielle des algorithmes de décodage de codes polaires à liste sont également proposés sur des architectures ARM. Comme attendu, les débits sont moindres, mais la consommation énergétique est réduite.

Dans le troisième chapitre, une architecture de processeur spécialisée pour le décodage de codes polaires est proposée. Le processeur fait partie de la famille des ASIP. Les outils de la société Tensilica sont utilisés. Les processeurs conçus à l'aide de ces outils sont des processeurs du type RISC. Le jeu d'instructions peut être étendu et la microarchitecture peut être modifiée afin de les rendre plus efficaces pour un type d'application. Ils conservent toutefois la versatilité des architectures RISC classiques et bénéficient d'un écosystème logiciel facilitant leur conception et le développement des programmes les ciblant. Tout comme les implémentations logicielles du deuxième chapitre, le programme est décrit dans un langage de haut niveau (C++). Les expérimentations et les mesures réalisées montrent que la spécialisation de l'architecture RSIC permet d'atteindre des débits comparables à ceux obtenus sur les architectures ARM, tout en réduisant la consommation énergétique d'un ordre de grandeur.

Dans le quatrième chapitre, une architecture de processeur du type TTA est conçue. Configurable plus finement et d'une structure plus modulaire, ce type d'architecture programmable se rapproche des implémentations matérielles dédiées, du point de vue de leur structure comme de celui de leurs performances. Tout comme les architectures du troisième chapitre, elles n'en restent pas moins versatiles et programmables. Elles sont également dotées d'un environnement de conception complet. Cet environnement est une suite logicielle libre développée par l'équipe \og Customized Parallel Computing \fg de l'université technique de Tampere (TUT). Contrairement aux architectures réalisées avec les outils de la société Tensilica, le modèle complet du processeur est produit par l'environnement, pour être utilisé par des logiciels de synthèse tiers. Deux architectures programmables sont proposées. La première supporte l'algorithme de décodage SC seul. La seconde supporte les algorithmes de décodage SC et SCAN. Les deux architectures programmables ont tout d'abord été synthétisées et implantées sur des circuits FPGA afin d'en effectuer la vérification fonctionnelle. Dans un second temps, des résultats de synthèse sur cible ASIC ont été générés pour la première architecture. Ces résultats montrent que cette architecture surpasse les débits obtenus sur les architectures x86, tout en réduisant la consommation énergétique de deux ordres de grandeur.

Dans le Tableau~\ref{tab:synthesis}, les caractéristiques des implémentations logicielles sur les différentes architectures de décodeurs considérées dans ce manuscrit sont récapitulées. Tout d'abord, les processeurs d'architecture x86 et ARM sont disponibles sur le marché. Au contraire, les ASIP sont des architectures spécialisées pour lequel des développements au niveau architectural doivent être engagés afin d'aboutir à une implantation sur cible ASIC. Cependant, les deux ASIP sont plus efficaces énergétiquement. L'architecture \texttt{TT-SC} est la plus performante du point de vue du débit et de l'efficacité énergétique. Néanmoins, il s'agit aussi de l'architecture la moins généraliste. Un des symptômes de ce manque de flexibilité est l'absence de système d'exploitation ciblant les architectures TTA. \`A l'inverse, les processeurs XTensa bénéficient d'un système d'exploitation fondé sur le noyau Linux. En effet, les processeurs XTensa sont basés sur des architectures RISC qui sont des architectures de processeurs classiques. En ce sens, leur flexibilité se rapproche de celles des architectures x86-64 et ARM.
L'ensemble des quatre architectures de processeurs offre des compromis marqués entre flexibilité, performance et efficacité énergétique.

\begin{table}[htp]
\centering
\caption{Existence, disponibilité d'un système d'exploitation, débits et consommations énergétiques des processeurs pour les différentes architectures considérées. Les intervalles de débits et de consommations énergétiques concernent le décodage de mots de codes polaires dont les tailles varient de $N=128$ à $N=1024$ et dont le rendement est $R=1/2$}
\label{tab:synthesis}
{\small\resizebox{\linewidth}{!}{
 \begin{tabular}{r|cccc}
  \multirow{2}{*}{\textbf{Architecture}}  & \multirow{2}{*}{\textbf{ASIP}}  & \textbf{Système}        & $\bm{\mathcal{T}_i}$  & $\bm{\mathcal{E}_b}$ \\
                                          &                                & \textbf{d'exploitation} & \textbf{Mb/s}         & \textbf{nJ}        \\
  \cmidrule(lr){1-1}
  \cmidrule(lr){2-2}
  \cmidrule(lr){3-3}
  \cmidrule(lr){4-4}
  \cmidrule(lr){5-5}
  
  x86-64 (Haswell)                        & \xmark              & \cmark                   & 120-260                 & 40-90              \\
  ARMv8-A                                 & \xmark              & \cmark                   & 30-40                   & 10-30              \\
  XTensa Polaire                          & \cmark              & \cmark                   & 30-70                   & 1-2                \\
  \texttt{TT-SC}                          & \cmark              & \xmark                   & 250-350                 & 0.1-0.2            \\
\end{tabular}
}}
\end{table}

\section*{Perspectives}

Les travaux présentés dans cette thèse ouvrent plusieurs pistes de recherche. 

Le décodeur logiciel proposé dans le deuxième chapitre est intégré à la suite logicielle libre AFF3CT de l'équipe CSN du laboratoire IMS de Bordeaux. Le but de ce projet est de proposer à la communauté scientifique des implémentations logicielles efficaces d'algorithmes de décodage de codes correcteurs d'erreurs. Les travaux réalisés dans le cadre de cette thèse l'ont grandement enrichi. Dans le futur, une plus grande variété d'algorithmes de décodage de codes polaires pourrait être supportée. Par exemple, les algorithmes SCF et SCS présentés dans le premier chapitre présentent un intérêt certain. Cependant, pour autant que nous le sachions, il n'existe aucune référence dans la littérature reportant les débits et les latences obtenus sur des architectures de processeur à usage général de ces deux algorithmes. Or, la faible complexité calculatoire de l'algorithme SCF semble indiquer de bonnes performances potentielles dans ce domaine, malgré des performances de décodage légèrement en retrait par rapport à celles de l'algorithme SCL. L'algorithme SCS présente quant à lui des performances de décodage égales à celles de l'algorithme SCL. Une étude récente tend à montrer que de hauts débits et de faibles latences pourraient être atteints~\cite{8351832}. Ils proposent également de nouvelles constructions de codes polaires appelés codes polaires multi-noyaux~\cite{7962750,8254147}. Leur principe est d'utiliser de nouveaux noyaux élémentaires afin de construire la matrice génératrice de codes polaires et non plus seulement les noyaux de taille 2 proposés originellement. L'intégration de ces codes dans AFF3CT et surtout la proposition d'implémentations logicielles efficaces de décodeurs logiciels associés présenteraient un grand intérêt pour la communauté scientifique.

Parmi les architectures programmables considérées dans ces travaux, celles permettant d'atteindre les meilleurs débits, latences et efficacités énergétiques sont les architectures TTA.
Un axe de recherche futur envisagé est la conception d'une architecture TTA pour le décodage des algorithmes de codes polaires à liste.
En effet, les performances de décodage sont bien plus élevées, au prix d'une plus grande complexité calculatoire.
Plusieurs étapes seront nécessaires. Tout d'abord, nos travaux montrent que le temps alloué à la propagation des sommes partielles doit être réduit.
Une solution pourrait être de s'inspirer de ce qui est réalisé dans les architectures matérielles dédiées.
Ensuite, l'architecture devra être étendue, afin de gérer le calcul des métriques de chemin et les LLR associés à chaque arbre de décodage. De plus, dans les algorithmes à liste, ces arbres de décodage doivent être parcouru en parallèle. Une méthode pour réaliser ces décodages parallèles pourrait être de concevoir une plate-forme multicœurs. La conception de telles plates-formes peut être facilitée grâce à la suite logicielle TCEMC~\cite{tcemc_2011}. Des architectures TTA multicœurs pour le décodage de turbo codes proposées dans la littérature~\cite{kultala_turbo_2013} pourraient être une bonne source d'inspiration.

Enfin, au cours de cette thèse, des travaux ont été réalisés afin d'intégrer les codes polaires dans des chaînes de communication plus complexes~\citemine{ghaffari_improving_2017,Ghaffari2018}. Le sujet de ces publications est l'implémentation de la technique d'accès multiple SCMA (Sparse Code Multiple Access). L'articulation de cette technique avec le codage et le décodage de codes polaires a été réalisé. Toutefois, par l'utilisation d'algorithmes de décodage de codes polaires à sorties souples, des échanges d'informations entre le décodeur polaire et le décodeur SCMA seraient possibles au sein d'un processus itératif, améliorant ainsi les performances de décodage de l'ensemble du système. Un autre cas d'application des codes polaires pourrait être l'implémentation de la couche physique de la norme 5G~\cite{3gpp_ts_2017}, sur des architectures de processeur à usage général comme sur des architectures TTA.
