%!TEX root = ../my_thesis.tex
\chapter*{Conclusions et perspectives}
\markboth{Conclusions et perspectives}{Conclusions et perspectives}
\addcontentsline{toc}{chapter}{Conclusions et perspectives}

\section*{Avantages et inconvénients de chaque approche}
\section*{Conclusions}
Dans le deuxième chapitre, des implémentations logicielles des algorithmes de codes polaires à liste sont proposées. L'originalité de ces implémentations tient à leur forte flexibilité. Cette flexibilité est inédite en comparaison des travaux publiés dans la littérature sur le même sujet. Tout d'abord, les données internes de l'algorithme sont représentées en virgule fixe, ce qui diminue l'empreinte mémoire et permet de réduire le temps nécessaire au décodage d'une trame. De plus, l'élagage de l'arbre de décodage est une simplification indispensable pour atteindre de hauts débits et de faibles latences de décodage. Cet élagage, dans les décodeurs proposés, est dynamiquement configurable. La flexibilité de l'élagage permet des compromis intéressants entre débit, latence et performance de décodage. Grâce à différentes optimisations, la flexibilité et la généricité du décodeur sont atteintes sans sacrifier le débit ou la latence de décodage. Une des versions de l'algorithme à liste, l'algorithme FASCL, permet même de dépasser le débit des implémentations de l'état de l'art sur les architectures x86. Des résultats d'implémentation des algorithmes de décodage de codes polaires à liste sont également proposés sur des architectures ARM. Les débits sont moindres, mais la consommation énergétique est réduite.

Dans le troisième chapitre, une architecture de processeur spécialisée pour le décodage de codes polaires est proposée. Le processeur fait partie de la famille des ASIP. Les outils de la marque Tensilica sont utilisés. Les processeurs conçus à l'aide des ces outils sont des processeurs de type RISC, dont le jeu d'instructions est étendu et la microarchitecture modifiée afin de les rendre plus efficaces pour une application donnée. Ils conservent toutefois la versatilité de architectures RISC classiques, et bénéficient d'un écosystème logiciel facilitant leur conception et le développement de programme les ciblant. Tout comme les implémentations du chapitre précédent, le programme est décrit logiciellement, dans un langage de haut niveau (C, C++, OpenCL). Les expérimentations et les mesures réalisées montrent que l'architecture de processeurs proposée permet d'atteindre des débits comparables à ceux obtenus sur les architectures ARM, tout en réduisant la consommation énergétique d'un ordre de grandeur.

Dans le quatrième chapitre, une architecture de processeur de type TTA est conçue. Configurable plus finement et d'une structure plus modulaire, ce type d'architecture se rapproche des implémentations matérielles dédiées, du point de vue de leur structure comme de celui de leurs performances. Tout comme les architectures du chapitre troisième, elles n'en restent pas moins versatiles et programmables. Elles sont également dotées d'un environnement de conception complet. Cet environnement est une suite logicielle libre développée par l'équipe \og Customized Parallel Computing \fg de l'université technique de Tampere (TUT). Contrairement aux architectures réalisées avec les outils de Tensilica, le modèle complet du processeur est produit par l'environnement, pour être utilisé par des logiciels de synthèse tiers. Deux architectures sont proposées. La première supporte l'algorithme SC seul. La seconde supporte les algorithmes SC et SCAN. Les deux sont synthétisées et implantées sur des circuits FPGA. Les résultats de synthèse de la première architecture montrent qu'elle permet de surpasser les débits obtenus sur les architectures x86, tout en réduisant la consommation énergétique de deux ordres de grandeur.

Dans le Tableau \ref{tab:synthesis}, les caractéristiques des différentes architectures de décodage considérées dans ce manuscrit sont présentées. Tout d'abord, les processeurs d'architecture x86-64 et ARM sont disponibles sur le marché. Au contraire, les ASIP sont des architectures spécialisées pour lequel des frais doivent être engagées afin d'en réaliser l'implantation sur cible ASIC. Cependant, les deux ASIP sont plus efficaces énergétiquement. L'architecture \texttt{TT-SC} est la plus performante du point de vue du débit et de l'efficacité énergétique. Néanmoins, il s'agit aussi de l'architecture la moins généraliste. Un des symptômes de ce manque de flexibilité et l'absence de système d'exploitation ciblant les architectures TTA, tandis que les processeurs XTensa bénéficient d'un système d'exploitation fondé sur le noyau Linux. En effet, les processeurs XTensa sont basés sur des architectures RISC qui sont des architectures de processeurs classiques. En ce sens, sa flexibilité se rapprochent de celles des architectures x86-64 et ARM.
L'ensemble des quatre architectures de processeurs offre des compromis marqués entre flexibilité, performance et efficacité énergétique.

\begin{table}[htp]
\centering
\caption{Existence, disponibilité d'un système d'exploitation, débits et consommation énergétiques des processeurs pour les différentes architectures considérées. Les intervalles de débit et de consommation énergétique concernent le décodage de mots de codes polaires dont les tailles varient de $N=128$ à $N=1024$ et dont le rendement est $R=1/2$}
\label{tab:synthesis}
{\small\resizebox{\linewidth}{!}{
 \begin{tabular}{r|cccc}
  \multirow{1}{*}{\textbf{Architecture}}  & \textbf{Processeur}  & \textbf{Système}        & $\bm{\mathcal{T}_i}$  & $\bm{\mathcal{E}_b}$ \\
                                          & \textbf{existant}    & \textbf{d'exploitation} & \textbf{Mb/s}         & \textbf{Mb/s}        \\
  \cmidrule(lr){1-1}
  \cmidrule(lr){2-2}
  \cmidrule(lr){3-3}
  \cmidrule(lr){4-4}
  \cmidrule(lr){5-5}
  
  x86-64 (Haswell)                        & \cmark              & \cmark                   & 120-260                 & 40-90              \\
  ARMv8-A                                 & \cmark              & \cmark                   & 30-40                   & 10-30              \\
  XTensa Polaire                          & \xmark              & \cmark                   & 30-70                   & 1-2                \\
  \texttt{TT-SC}                          & \xmark              & \xmark                   & 250-350                 & 0.1-0.2            \\
\end{tabular}
}}
\end{table}

%\multirow{1}{*}{\textbf{Débit}} 
\section*{Perspectives}

Le décodeur logiciel proposé dans le chapitre \ref{chap:soft_scl} fait partie du projet AFF3CT de l'équipe CSN du laboratoire IMS de Bordeaux. Le but de ce projet est de proposer à la communcauté scientifique des implémentations logicielles efficace d'algorithmes de décodage de codes correcteurs d'erreurs. Les travaux réalisés dans le cadre de cette thèse l'ont grandement enrichi. Dans le futur, une plus grande variété d'algorithmes de décodage de codes polaires pourrait être supportée. Par exemple, les algorithmes SCF et SCS présentés dans le chapitre \ref{chap:soft_scl} présentent un intérêt certains. Cependant, pour autant que nous le sachions, il n'existe aucune référence dans la littérature reportant les débits et les latences obtenus sur des architectures de processeur à usage général. Or, la faible complexité calculatoire de l'algorithme SCF semble indiquer de bonnes performance potentielles dans ce domaine, malgré des performances de décodage légèrement en retrait de l'algortihme SCL. L'algorithme SCS présente quant à lui des performances de décodage égales à celles de l'algorithme SCL. Des travaux récents tendent à montrer que de hauts débit et de faibles latences pourraient être atteints \cite{8351832}.

Parmi les architectures considérés dans ces travaux, celles permettant d'atteindre les meilleurs débits, latences et efficacités énergétiques sont les architectures TTA. Un axe de recherche envisagé est la conception d'une architecture TTA pour le décodage des algorithmes de codes polaires à liste. En effet, les performances de décodage sont bien plus élevées, au prix d'une plus grande complexité calculatoire. Plusieurs étapes seront nécessaires. Tout d'abord, nos travaux montrent que le temps alloué à la propagation des sommes partielles doit être réduit. Une solution pourrait être de s'inspirer de ce qui est réalisé dans les architectures matérielles dédiées. Ensuite l'architecture devra être étendue, afin de permettre le parcours de plusieurs arbres de décodage en parallèle, ainsi que la gestion des métriques de chemin. La flexibilité de l'architecture pourra alors être mise à profit afin de gérer un large panel de codes polaire saux caractéristiques variées, tels que ceux définis dans le standard 5G.

Enfin, la suite logicielle TCE facilite la conception de plateformes multicœurs. Des travaux récents montrent les gains potentiels de ces architectures en termes de débit pour des turbo codes \cite{}. Des problématiques de dimensionnement du nombre de cœurs, des interfaces et des mémoires doivent être relevées pour créer de tels systèmes.