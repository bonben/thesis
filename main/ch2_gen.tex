%!TEX root = ../gen_code.tex
\ifthenelse{\equal{\templateVersion}{UBX}}
{
\chapter{Décodeur logiciel de codes polaires à liste}
\label{chap:soft_scl}
}
{
\Chapter{D\'ECODEUR LOGICIEL DE CODES POLAIRES \`A LISTE}
\label{chap:soft_scl}
}


\begin{figure}[htp]
\lstset{language=C++,
        basicstyle=\footnotesize\ttfamily,
        keywordstyle=\bfseries\color{green!40!black},
        commentstyle=\itshape\color{purple!40!black},
        identifierstyle=\color{blue},
        stringstyle=\color{orange},
        morecomment=[l][\color{magenta}]{\#}
}
\begin{lstlisting}[language=C++, numbers=none, tabsize=2, basicstyle=\footnotesize\ttfamily]
  decode(node)
  {
    f(node);
    decode(node_gauche);
    g(node);
    decode(node_droit);
    if(node.frozen)
      node.s = 0;
    else
      node.s = R1(node);
  }
  \end{lstlisting}
  \caption{Implémentation logicielle des fonctions $f$ et $g$ utilisant la bibliothèque MIPP.}
  \label{fig:mipp}
\end{figure}
\begin{figure}[htp]
\lstset{language=C++,
        basicstyle=\footnotesize\ttfamily,
        keywordstyle=\bfseries\color{green!40!black},
        commentstyle=\itshape\color{purple!40!black},
        identifierstyle=\color{blue},
        stringstyle=\color{orange},
        morecomment=[l][\color{magenta}]{\#}
}
\begin{lstlisting}[language=C++, numbers=none, tabsize=2, basicstyle=\footnotesize\ttfamily]
  decode()
  {
    f(n1);
    n2.s = 0;
    g(n1);
    n3.s = R1(node);
  }
  \end{lstlisting}
  \caption{Implémentation logicielle des fonctions $f$ et $g$ utilisant la bibliothèque MIPP.}
  \label{fig:mipp}
\end{figure}

\begin{figure}[htp]
\lstset{language=C++,
        basicstyle=\footnotesize\ttfamily,
        keywordstyle=\bfseries\color{green!40!black},
        commentstyle=\itshape\color{purple!40!black},
        identifierstyle=\color{blue},
        stringstyle=\color{orange},
        morecomment=[l][\color{magenta}]{\#}
}
\begin{lstlisting}[language=C++, numbers=none, tabsize=2, basicstyle=\footnotesize\ttfamily]
  decode()
  {
    f(n1);
    n2.s = 0;
    g(n1);
    n3.s = R1(node);
  }
  \end{lstlisting}
  \caption{Implémentation logicielle des fonctions $f$ et $g$ utilisant la bibliothèque MIPP.}
  \label{fig:mipp}
\end{figure}
