%!TEX root = ../my_thesis.tex

\chapter{Les codes polaires}

Résumé

\vspace*{\fill}
\minitocTITI
\vspace*{\fill}

\subsection*{Introduction}

\section{Codes correcteurs d'erreurs}

\subsection{Introduction}
\begin{itemize}
\item 
\end{itemize}
\subsection{Encodage du code polaire}
\begin{itemize}
\item 
\end{itemize}
\subsection{Noeuds de parité et d'égalité}
\begin{itemize}
\item 
\end{itemize}
\subsection{Polarisation}
\begin{itemize}
\item 
\end{itemize}

\section{Classes d'algorithmes de décodage}

\subsection{Annulation Successive}
\begin{itemize}
\item 
\end{itemize}
\subsection{Annulation Successive Liste}
\begin{itemize}
\item 
\end{itemize}
\subsection{Algorithmes itératifs à sortie souple}
\begin{itemize}
\item 
\end{itemize}

\section{Améliorations algorithmiques}

\subsection{Elagage de l'arbre de décodage}

\subsubsection{Fast SC}
\begin{itemize}
\item 
\end{itemize}
\subsubsection{Fast SCL}
\begin{itemize}
\item 
\end{itemize}
\subsubsection{Fast SCAN}
\begin{itemize}
\item 
\end{itemize}

\subsection{Adaptive}

\section{Codes polaires de taille variables}

\subsection{Techniques par poinçonnage}
\begin{itemize}
\item 
\end{itemize}
\subsection{Techniques par raccourcissement}
\begin{itemize}
\item 
\end{itemize}

\subsection*{Conclusion}
