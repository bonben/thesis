%!TEX root = ../my_thesis.tex

\chapter{Les codes polaires}

Résumé

\vspace*{\fill}
\minitocTITI
\vspace*{\fill}

\subsection*{Introduction}

\section{Principe et construction}

\subsection{Contexte}
\begin{itemize}
\item Chaine de communication : complet -> simplifié data -> encodeur -> canal -> décodeur-data
\item Introduire redondance
\item Codes linéaires - codes en blocs : évocation de certains pour exemple (hamming ? ldpc ?)
\item Info souple / dure

\end{itemize}
\subsection{Noeuds de parité et d'égalité}
\begin{itemize}
\item Codes bloc : réseau parité et égalité (reprendre exemple)
\item Dans codes polaires, noyau 2 utilisé
\item donner les équations
\end{itemize}
\subsection{Polarisation}
\begin{itemize}
\item donner valeurs pour bec
\item simulations, représentation de la polarisation
\end{itemize}
\subsection{Constructions des codes polaires}
\begin{itemize}
\item puisque polarisation -> bits gelés
\item equations SC
\end{itemize}

\section{Les algorithmes de décodage de codes polaires}

\subsection{Annulation Successive}
\begin{itemize}
\item Séquencement
\item Parallélisme
\end{itemize}
\subsection{Annulation Successive Liste}
\begin{itemize}
\item Algorithme
\item Calcul Métrique

\end{itemize}
\subsection{Algorithmes itératifs à sortie souple}
\begin{itemize}
\item Noeud 2 à sortie souple
\item BP
\item SCAN
\end{itemize}

\subsection{Autres algorithmes}
\begin{itemize}
	\item SCS
	\item SCF
\end{itemize}
\section{Améliorations algorithmiques}

\subsection{Elagage de l'arbre de décodage}

\subsubsection{Fast SC}
\begin{itemize}
\item R0 - R1
\item SPC - REP
\item détailler calculs évitables (g0, f0, grep)
\end{itemize}
\subsubsection{Fast SCL}
\begin{itemize}
\item détailler adaptation élagage pour SC Liste - calculs métriques
\item discussion chase spc - r1 ?

\end{itemize}
\subsubsection{Fast SCAN}
\begin{itemize}
\item 
\end{itemize}

\subsection{Adaptive}

\section{Codes polaires de taille variables}
% https://arxiv.org/pdf/1701.06458.pdf 4-5-6-7-8-9


\subsection*{Conclusion}
