%!TEX root = ../my_thesis.tex
\chapter*{Introduction}
\markboth{Introduction}{Introduction}
\addcontentsline{toc}{chapter}{Introduction}

Le décodage de codes polaires sur des architectures programmable constitue le sujet de cette thèse. Deux axes principaux sont développés. Le premier est l'implémentation des algorithmes de décodage sur des architectures de processeurs existantes. Le deuxième est la conception d'architectures programmables spécialisées dans le décodage de codes polaires. En introduction, le contexte général des codes correcteurs d'erreurs est établi puis le sujet de recherche est présenté et motivé. La structure du manuscrit est ensuite détaillée avant une liste des contributions contenues dans ces travaux de thèse.

\section*{Contexte}
La chaîne de communication la plus élémentaire est constituée d'une source, d'un canal de communication et d'un destinataire. Un message est émis par la source à travers le canal jusqu'au destinataire. \`A moins d'un canal idéal, des perturbations peuvent détériorer le message, et des erreurs peuvent se produire. Ces erreurs peuvent être corrigées à l'aide de codes correcteurs d'erreurs. Il s'agit, en aval de la source, d'ajouter de la redondance au message d'origine. Cette redondance est utilisée, en amont du destinataire, afin de corriger les erreurs potentielles.

Les codes polaires sont une famille de codes correcteurs d'erreurs, inventés en 2008 par Erdal Ar{\i}kan. Leur sélection pour faire partie de la prochaine norme 5G atteste de leur efficacité et montre la nécessité de concevoir des systèmes performants pour en réaliser les deux étapes : l'encodage et le décodage. Jusqu'à présent, dans les systèmes de communications mobiles, le décodage des codes correcteurs d'erreurs est réalisé à l'aide d'architectures matérielles dédiées. Ces architectures sont très efficaces en termes de débit, de latence et de consommation énergétique. Cependant, une alternative est possible : la description logicielle des algorithmes de décodage et leur exécution sur des processeurs généralistes. Cette alternative offre plusieurs avantages. Elle permet une plus grande flexibilité du réseau, capable d'adapter l'effort de calcul à la demande, dans l'espace, c'est à dire entre les différents \noeuds du réseau, et à travers le temps. Elle permet également une meilleure évolutivité, puisqu'un même circuit peut être utilisé pour supporter des améliorations ou de nouvelles versions des algorithmes de décodage. Enfin, il est plus aisé de distribuer le calcul, pour s'approcher d'une infrastructure de type Cloud. Ces travaux ont pour but d'étudier et de proposer des solutions adéquates : le décodage de codes polaires sur des architectures programmables.

\section*{Structure du manuscrit de thèse}

Le premier chapitre présente les codes polaires. Après une brève introduction portant sur la composition d'une chaîne de communication numérique, les codes polaires et le principe de polarisation sont présentés. Ensuite, les différents algorithmes de décodage de codes polaires sont détaillés. Enfin, les méthodes d'élagage de l'arbre, qui constituent des simplifications algorithmiques des algorithmes de décodage, sont présentées.

Un décodeur générique et flexible de codes polaires est proposé dans le chapitre \ref{chap:soft_scl}. Il implémente les algorithmes de décodage à liste. Les concepts de généricité et de flexibilité du décodeur sont définis. La \textit{généricité} du décodeur est sa capacité à supporter une très grande variété de codes polaires, en termes de taille de mot de code, de construction et de concaténation avec des codes détecteurs d'erreurs. Ils peuvent donc s'adapter à un très grand nombre de cas de figure. La \textit{flexibilité} est définie comme sa capacité à configurer l'algorithme de décodage dans le but de proposer des compromis entre débit, latence et performance de décodage. Ce décodeur générique et flexible, grâce à plusieurs améliorations algorithmiques, approche, atteint voire dépasse les décodeurs logiciels de l'état de l'art, selon les paramètres expérimentaux. Ce décodeur est intégré au sein d'un logiciel libre, le code source est donc publié. Ceci permet son utilisation par l'ensemble de la communauté, ainsi que de reproduire l'intégralité des résultats présentés.

Un défaut de ce type de décodeurs logiciels est toutefois l'énergie consommée. En effet les processeurs considérés sont très généraliste, et cela se paie par d'importantes complexité matérielle et consommation énergétique. La deuxième partie de nos travaux a pour but de proposer deux architectures programmables à faible consommation et haute performance pour le décodage de codes polaires. Le but de ces travaux est de conserver des architectures flexibles, tout en améliorant l'efficacité énergétique. Les chapitres ~\ref{chap:tensilica} et~\ref{chap:tta} empruntent deux méthodologies de conception différentes. Par conséquent, la structure des architectures résultantes ainsi que leurs performances sont radicalement différentes.

Dans le chapitre ~\ref{chap:tensilica}, la méthodologie considérée est celle de la spécialisation d'une architecture de base. Cette spécialisation est effectuée à l'aide des outils de la marque Tensilica. Elle consiste à configurer une architecture de processeur de base et à étendre son jeu d'instructions par l'ajout d'unités matérielles dédiées. Les processeurs ainsi conçus atteignent des débits comparables à ceux obtenus sur les processeurs d'architecture ARM, tout en réduisant la consommation énergétique d'un ordre de grandeur. 

Dans le chapitre \ref{chap:tta}, la méthode de conception permet une configuration plus fine du processeur. Sa structure ainsi que les modèles matériels des unités élémentaires du processeurs sont complètement définis par l'utilisateur. L'architecture obtenue permet d'atteindre de très hauts débits et de faibles latences tout en réduisant fortement la complexité matérielle et donc la puissance dissipée. L'architecture proposée fait partie de la famille des TTA (Transport Triggered Architecture). Les résultats d'implémentations montrent que les décodeurs proposés dépassent le débits des implémentations logicielles sur architecture x86, et réduisent la consommation énergétique de deux ordres de grandeur.

\clearpage

\section*{Contributions des travaux de thèse}
Les différentes contributions originales de ces travaux de thèse sont : 
\begin{enumerate}
	\item La proposition d'un décodeur logiciel des algorithmes de décodage de codes polaires à liste. Les caractéristiques principales de ce décodeur logiciel sont sa généricité et sa flexibilité. Les débits obtenus dépassent les débits des implémentations logicielles de la littérature sur des architectures de processeurs x86. Des résultats d'exécution sur architecture ARM sont également présentés, constituant la première référence de la littérature. Ceci est détaillé dans le deuxième chapitre.
  	\item La proposition de la première architecture ASIP spécialisée dans le décodage de codes polaires. Cette architecture permet de conserver un grande flexibilité tout en améliorant l'efficacité énergétique. Les débits obtenus sont comparables à ceux de l'architecture ARM. La consommation énergétique est réduite d'un ordre de grandeur. Ceci est présenté dans le troisième chapitre.
  	\item La proposition de la première architecture de type TTA (Transport Triggered Architecture) pour le décodage de codes polaires. Il s'agit d'un type spécifique d'ASIP. Tout comme les architectures précédentes, elle bénéficie d'une grande flexibilité. Elle permet un important parallélisme d'instruction et une définition fine de sa struture. Ainsi, les débits obtenus dépassent ceux obtenus sur les architectures de processeur x86 et la consommation énergétique est quant à elle réduite de deux ordres de grandeur. Ceci est détaillé dans le quatrième chapitre.

\end{enumerate}

\vspace{1cm}
Ces différentes contributions ont fait l'objet de publications scientifiques : \\

\begin{itemize}
		\item[$\bullet$] Conférences nationales :
	\begin{itemize}
		\item[$\bullet$] M. Léonardon, C. Leroux, and C. Jégo, “Les codes polaires, algorithmes et décodeurs.”
		CNES CCT TSI Technologies pour la 5G - segment spatial, 2016.
     	\item[$\bullet$] A. Cassagne, M. Léonardon, O. Hartmann, T. Tonnellier, G. Delbergue, V. Giraud, C. Leroux, R. Tajan, B. Le Gal, C. Jégo, O. Aumage, and D. Barthou, “AFF3CT : Un environnement de simulation pour le codage de canal,” GdR SoC2, 2017.
	\end{itemize}
	\item[$\bullet$] Conférences internationales :
	\begin{itemize}
		\item[$\bullet$]  A. Cassagne, M. Léonardon, O. Hartmann, G. Delbergue, T. Tonnellier, R. Tajan, C. Leroux, C. Jégo, B. Le Gal, O. Aumage, and D. Barthou, “Fast simulation and prototyping with AFF3CT,” in IEEE International Workshop on Signal Processing Systems (SiPS), 2017.
		\item[$\bullet$] M. Léonardon, C. Leroux, D. Binet, J. M. P. Langlois, C. Jégo, and Y. Savaria, “Custom low power processor for polar decoding,” in IEEE International Symposium on Circuits and Systems (ISCAS), 2018.
		\item[$\bullet$] M. Léonardon, C. Leroux, P. Jääskeläinen, C. Jégo, and Y. Savaria, “Transport triggered polar decoders,” in IEEE International Symposium on Turbo Codes and Iterative Information Processing (ISTC), 2018.
		\item[$\bullet$] A. Ghaffari, M. Léonardon, Y. Savaria, C. Jégo, and C. Leroux, “Improving performance of SCMA MPA decoders using estimation of conditional probabilities,” in IEEE International New Circuits and Systems Conference (NEWCAS), 2017.
	\end{itemize}
	\item[$\bullet$] Revues internationales avec comité de lecture :
	\begin{itemize}
		\item[$\bullet$] M. Léonardon, A. Cassagne, C. Leroux, C. Jégo, L.-P. Hamelin, and Y. Savaria, “Fast
        and flexible software polar list decoders,” en cours de revue, Springer Journal of Signal Processing Systems
        (JSPS), 2018.
        \item[$\bullet$] A. Ghaffari, M. Léonardon, A. Cassagne, C. Leroux, and Y. Savaria, “Toward high performance implementation of 5G SCMA algorithms,” soumis à IEEE Access, 2018.
	\end{itemize}
\end{itemize}

