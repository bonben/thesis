%!TEX root = ../my_thesis.tex
\chapter*{Introduction}
\markboth{Introduction}{Introduction}
\addcontentsline{toc}{chapter}{Introduction}

Le décodage de codes polaires sur des architectures programmable constitue le sujet de cette thèse. Deux axes principaux sont développés. Le premier est l'implémentation des algorithmes de décodage sur des architectures de processeurs existantes. Le deuxième est la conception d'architectures programmables spécialisées dans le décodage de codes polaires. En introduction, le contexte général des codes correcteurs d'erreurs est établi puis le sujet de recherche est présenté et motivé. La structure du manuscrit est ensuite détaillée avant une liste des contributions contenues dans ces travaux de thèse.

\section*{Contexte}

La chaîne de communication la plus élémentaire est constituée d'une source, d'un canal de communication et d'un destinataire. Un message est émis par la source à travers le canal jusqu'au destinataire. \`A moins d'un canal idéal, des perturbations peuvent détériorer le message, et des erreurs peuvent se produire. Ces erreurs peuvent être corrigées à l'aide de codes correcteurs d'erreurs. Il s'agit, en aval de la source, d'ajouter de la redondance au message d'origine. Cette redondance est utilisée, en amont du destinataire, afin de corriger les erreurs potentielles.

Claude Shannon propose en 1948 une formalisation mathématique de l'information \cite{shannon_mathematical_2001}. Un des résultats majeurs de ses travaux est l’existence d'une limite théorique à la quantité d'information transmissible sur un canal de communication. Cette limite doit être atteinte à l'aide d'un code correcteur d'erreurs dont la construction reste à déterminer.
Dès lors, la communauté scientifique s'empare de cette problématique.
Parallèlement, John Bardeen, Walter Brattain et William Shockley inventent le transistor. Les circuits électronique vont constituer l'outil nécessaire à la réalisation techniques de codes correcteurs d'erreurs. Ainsi, de 1948 à aujourd'hui, les chercheurs et ingénieurs mettent au point des algorithmes et des architectures de systèmes de correction d'erreur de plus en plus complexe, permettant de s'approcher de la limite théorique établie par Claude Shannon.

Les codes polaires sont une des familles de codes correcteurs d'erreurs les plus récentes. Ils solutions inventés en 2008 par Erdal Ar{\i}kan \cite{arikan_channel_2009}. Leur sélection pour faire partie de la prochaine norme 5G atteste de leur efficacité et montre la nécessité de concevoir des systèmes performants pour en réaliser les deux étapes : l'encodage et le décodage. Jusqu'à présent, dans les systèmes de communications mobiles, le décodage des codes correcteurs d'erreurs est réalisé à l'aide d'architectures matérielles dédiées. Ces architectures sont très efficaces en termes de débit, de latence et de consommation énergétique. Elles souffrent cependant d'un manque de flexibilité.


La description logicielle des algorithmes de décodage et leur exécution sur des processeurs généralistes apparaît comme une alternative prometteuse palliant ce manque de flexibilité. Des travaux récents montrent que de telles implémentations permettent d'atteindre de hauts débits et de faibles latences \cite{sarkis_fast_2014,giard_fast_2014}. Cette flexibilité se traduit par une meilleure évolutivité des systèmes et une intégration facilitée. Il est plus aisé de distribuer le calcul, pour s'approcher d'une infrastructure de type Cloud. Les travaux décrits dans cette thèse ont pour but d'étudier et de proposer des solutions adéquates : le décodage de codes polaires sur des architectures programmables.

\section*{Structure du manuscrit de thèse}

les algorithmes de décodage à liste. Nous proposons une implémentation dont les caractéristiques principales sont la généricité et la flexibilité. La généricité, d'une part, permet aux décodeurs proposé de supporter une très grande variété de codes polaires, en termes de taille de mot de code, de construction et de concaténation avec des codes détecteurs d'erreurs. Ils peuvent donc s'adapter à un très grand nombre de cas de figure. \'Etant intégré au sein d'un logiciel libre, ces décodeurs sont publiés et disponibles pour la communauté. La flexibilité, d'autre part, apporte des compromis inédits entre les performances de décodage, le débit et la latence. Différentes optimisation leurs permettent également d'atteindre et de dépasser les débits de implémentations de l'état de l'art. Des résultats d'exécution sont également reportés pour des architectures ARM, constituant ainsi la première référence du genre dans la littérature.

Un défaut de ces implémentations est toutefois l'énergie consommée. En effet les processeurs considérés sont très généraliste, et cela se paie par une très grande complexité matérielle et une consommation énergétique importante. La deuxième partie de nos travaux a pour but de proposer deux architectures programmables à faible consommation et haute performance pour le décodage de codes polaires. Deux méthodes distinctes sont utilisées pour concevoir et générer de telles architectures.

Le troisième chapitre est consacré à la première méthode de conception. Elle consiste à spécialiser une architecture de processeur de base, en la configurant et en étendant son jeu d'instructions, par l'ajout d'unités matérielles dédiées. Ceci est réalisé à l'aide des outils de conception de Tensilica. Les processeurs ainsi conçus atteignent des débits comparables à ceux obtenus sur les processeurs d'architecture ARM, tout en réduisant la consommation énergétique d'un ordre de grandeur. Ils restent cependant très versatiles. Le compilateur et les outils de débogage, profilage et simulation s'adaptent et sont capables de gérer l'architecture modifiée et étendue.

Le quatrième chapitre traite de la seconde méthode de conception qui a permis la proposition de nouvelles architectures de processeurs spécialisés. Cette méthode permet de définir la structure du processeur et le modèle matériel de chacune de ses unités fonctionnelles. Ainsi, l'architecture est très finement conçue, afin d'obtenir de très hauts débits et de faibles latences tout en réduisant fortement la complexité matérielle et donc la puissance dissipée. L'architecture fait partie de la famille des TTA (Transport Triggered Architecture). Une suite logicielle libre permet d'automatiser une grande partie du processus de conception. Le compilateurs et les autres outils de conception, comme précédemment, sont capables de s'adapter aux changements d'architecture. Les résultats d'implémentations montrent que les décodeurs proposés dépassent le débits des implémentations logicielles sur architecture x86, et réduisent la consommation énergétique de deux ordres de grandeur.

\clearpage

\section*{Contributions des travaux de thèse}
Les différentes contributions originales de ces travaux de thèse sont : 
\begin{enumerate}
	\item La proposition d'implémentations logicielles des algorithmes de décodage de codes polaires à liste. Leur originalité en regard des autres implémentations de la littérature tient dans leur généricité et leur flexibilité. Au contraire des implémentations des algorithmes à liste logicielles précédentes, celles-ci sont les premières à permettre une représentation en virgule fixe sur une nombre réduit des données internes de l'algorithme. Des techniques complémentaires de stockage de ces données internes sont explorées et comparées. Une nouvelle méthode de tri est appliquée sur des portions de l'algorithme. Les débits obtenus dépassent les débits des implémentations logicielles de la littérature sur des architectures de processeurs x86. Les décodeurs sont également exécuté sur des processeurs d’architecture ARM et les résultats sont reportés, pour la première fois dans la littérature. Ceci est détaillé dans le deuxième chapitre.
  	\item La proposition d'une architecture de processeur spécialisé dans le décodage de codes polaires. Il s'agit d'un processeur à jeu d'instructions spécifique à l'application (ASIP : Application Specific Instruction set Processor). Le programme exécuté sur ces processeurs est décrit logiciellement. Les débits obtenus sont comparables à ceux de l'architecture ARM. La consommation énergétique est réduite d'un ordre de grandeur. Ceci est présenté dans le troisième chapitre.
  	\item Une nouvelle architecture d'ASIP est proposée. Elle fait partie de la famille des architectures déclenchées par le transport (TTA : Transport Triggered Architecture). Les débits  dépassent ceux obtenus sur les architectures de processeurs x86. La consommation énergétique est quant à elle réduite de deux ordres de grandeur. Ceci est détaillé dans le quatrième chapitre.

\end{enumerate}


Ces différentes contributions ont fait l'objet de publications scientifiques : \\

\begin{itemize}
		\item Communication nationale sans acte :
	\begin{itemize}
		\item M. Léonardon, C. Leroux, and C. Jégo, “Les codes polaires, algorithmes et décodeurs.”
		CNES CCT TSI Technologies pour la 5G - segment spatial, 2016.
	\end{itemize}
	\item Communication nationale avec acte :
	\begin{itemize}
     	\item A. Cassagne, M. Léonardon, O. Hartmann, T. Tonnellier, G. Delbergue, V. Giraud,
			  C. Leroux, R. Tajan, B. Le Gal, C. Jégo, O. Aumage, and D. Barthou, “AFF3CT :
			  Un environnement de simulation pour le codage de canal,” GdR SoC2, 2017.
	\end{itemize}
	\item Communication internationale sans acte :
	\begin{itemize}
		\item  A. Cassagne, M. Léonardon, O. Hartmann, G. Delbergue, T. Tonnellier, R. Tajan,
			   C. Leroux, C. Jégo, B. Le Gal, O. Aumage, and D. Barthou, “Fast simulation and
			   prototyping with AFF3CT,” in IEEE International Workshop on Signal Processing Systems (SiPS), 2017.
	\end{itemize}
	\item Communications internationales avec actes :
	\begin{itemize}
		\item M. Léonardon, C. Leroux, D. Binet, J. M. P. Langlois, C. Jégo, and Y. Savaria,
		“Custom low power processor for polar decoding,” in IEEE International
		Symposium on Circuits and Systems (ISCAS), 2018.
		\item M. Léonardon, C. Leroux, P. Jääskeläinen, C. Jégo, and Y. Savaria, “Transport
		triggered polar decoders,” in IEEE International Symposium on Turbo Codes and
		Iterative Information Processing (ISTC), 2018.
		\item A. Ghaffari, M. Léonardon, Y. Savaria, C. Jégo, and C. Leroux, “Improving performance of SCMA MPA decoders using estimation of conditional probabilities,” in IEEE 
		International New Circuits and Systems Conference (NEWCAS), 2017.
	\end{itemize}
	\item Communications dans des revues internationales avec comité de lecture :
	\begin{itemize}
		\item M. Léonardon, A. Cassagne, C. Leroux, C. Jégo, L.-P. Hamelin, and Y. Savaria, “Fast
        and flexible software polar list decoders,” en cours de revue, Springer Journal of Signal Processing Systems
        (JSPS), 2018.
        \item A. Ghaffari, M. Léonardon, A. Cassagne, C. Leroux, and Y. Savaria, “Toward high
        performance implementation of 5G SCMA algorithms,” soumis à IEEE Access, 2018.
	\end{itemize}
\end{itemize}

