%!TEX root = ../my_thesis.tex
\chapter*{Introduction}
\markboth{Introduction}{Introduction}
\addcontentsline{toc}{chapter}{Introduction}

Les travaux développés dans le cadre de cette thèse portent sur la proposition de méthodes et la définition 
d'architectures associées permettant 
l'amélioration des performances de décodage des turbo codes. Tout d'abord le contexte des codes correcteurs d'erreurs 
est dressé. Ceci mène à la problématique traitée tout au long de ce manuscrit. L'agencement du manuscrit est 
ensuite détaillé. Enfin, les contributions réalisées au cours des travaux de thèse sont énoncés.

\section*{Contexte et problématique}
L'année 2016 marque le centenaire de la naissance de Claude Shannon. Ingénieur en génie électrique et mathématicien, il 
est le père de la théorie de l'information. L'origine de cette discipline -- se plaçant à l'intersection des mathématiques, 
des statistiques, du traitement du signal et de l'électronique -- remonte à la publication de l'article fondateur 
"A mathematical Theory of Communication" \cite{shannon_mathematical_2001}, paru en 1948 dans la revue interne des laboratoires Bell. Quelques mois 
auparavant, au même endroit, le transistor est inventé par John Bardeen, Walter Brattain et William Shockley. Ces deux 
avancées scientifiques, concomitantes, ont en moins de 70 ans amplement modifié nos sociétés et leurs manières d’interagir entre elles. 
% En effet, d'une part, ce semi-conducteur permet de représenter par son état de conduction (ouvert ou fermé) les deux états d'une
% information binaire. D'autre part l'unité de mesure de l'apport de d'une information, le bit, correspond à cette levée 
% d'incertitude lors de la réponse vrai ou faux.

Via la formalisation de l'information de manière abstraite et mathématique, le sens d'un message n'importe plus. Il est 
désormais possible de mesurer et de quantifier l'information. %En découle la possibilité de théoriser la 
Dans son article, Shannon montre que chaque canal de transmission peut faire transiter une quantité maximale 
d'information de manière fiable. De fait, en raison du bruit inhérent aux canaux de transmission, des erreurs peuvent apparaître
lors de la réception du message. Afin de palier à cela, Shannon énonce qu'il existe un système de codage correcteur d'erreurs, 
basé sur l'insertion d'informations redondantes à l’émission, permettant de corriger complètement les effets des distorsions et ainsi, de reconstituer parfaitement
le contenu du message émis. Cependant, Shannon ne fourni aucune indication quant à la façon de concevoir un tel code
correcteur d'erreurs.

Face à l'enjeu que représente la transmission fiable de l'information, la communauté scientifique s'est consacrée à la 
construction de tels codes. Tout d'abord, en 1950, Richard Hamming invente un code éponyme permettant de corriger tous 
les messages contenant une unique erreur binaire \cite{hamming}. S'en est suivi 15 années de découvertes successives avec les codes de Reed-Muller \cite{reedmuller1,reedmuller2} en 1954,
puis les codes convolutifs par Peter Elias en 1955 \cite{elias}, les codes de BCH en 1959 \cite{hocquenghem1959codes,bose1960class}, les codes LDPC par Robert Gallager en 1962 \cite{gallager1962low}.
Ces différentes familles de codes correcteurs d'erreurs permettent de corriger de 1 à $N$ erreurs de transmission dans un message en fonction de la quantité d'information redondante ajoutée lors de l'émission du message.
Enfin, en 1966, David Forney introduit le principe de la concaténation de codes correcteurs d'erreurs \cite{forney1966concatenated}. Cependant, à partir de 1966, les avancées
sur la construction de codes correcteurs d'erreurs se font rare. Cela fait alors dire à Robert McEliece en 1971 que 
\og Le codage de canal est mort \fg\cite{codingdead}. 

C'était sans compter sur la découverte deux décennies plus tard des turbo codes par Claude Berrou et Alain Glavieux \cite{berrouTC}. 
Grâce à leurs performances, cette nouvelle famille de codes correcteurs d'erreurs se place en rupture avec les
schémas de codage alors existants. Dès leur introduction, les turbo codes sont employés dans différents 
standards de communications numériques adressant des contextes applicatifs divers et variés. L'une des métriques de 
performance d'un code correcteur d'erreurs est sa capacité de correction en fonction de la qualité de transmission. 
Une telle courbe de performance pour un turbo code est divisé en deux parties. Dans la première, nommée région de 
convergence, un faible incrément de la qualité de transmission résulte en une amélioration importante des performances de
décodage. En revanche, dans la région du plancher d'erreurs, l'augmentation de la qualité de transmission ne résulte qu'en
une amélioration marginale des performances de décodage. Cette région est alors particulièrement limitante pour des applications
nécessitant de très faibles taux d'erreurs.

Par exemple, 
pour le stockage de masse, les taux d'erreurs cibles sont de plus en plus critiques et atteignent actuellement $10^{-18}$.
Dans un contexte de 
lien optique en espace libre, correspondant à des communications satellitaires par faisceau optique, un 
plancher d'erreurs situé sous les $10^{-9}$ est recherché. Le même ordre de grandeur est attendu pour le standard 
CCSDS-2. Enfin les applications de télémesure ou de 
contrôle-commande d'aéronef sans humain à bord nécessitent elles aussi des taux d'erreurs particulièrement faibles. Dès 
lors, une amélioration des performances de décodage, et notamment la réduction du plancher d'erreurs est primordiale.

D'autre part, lorsqu'un turbo code est retenu pour un standard de communications numériques, ses performances sont adaptées aux
contraintes applicatives correspondant au cas d'emploi de ce standard. Cependant, les besoins applicatifs évoluent au cours du temps et peuvent diverger de ceux originellement considérés. Ainsi, des exigences de plus faibles taux 
d'erreur pour une qualité de transmission constante peuvent apparaître. Deux solutions sont alors envisageables. La première
consiste à repenser le code correcteur d'erreur afin d'atteindre ces nouveaux besoins. Néanmoins, cette approche est 
particulièrement coûteuse. De fait, l'ensemble de l’infrastructure devient obsolète et un nouveau déploiement 
d'équipements, que ce soit pour la transmission ou la réception, s'avère nécessaire. Cette solution est par exemple 
difficilement envisageable pour un contexte de communications satellitaires.

Une autre solution consiste à modifier uniquement la partie réception. Dans ce contexte, les modifications des performances 
se situent au niveau des fonctions de réception. Ainsi, ces dernières années, la communauté scientifique a proposé différentes 
approches permettant d'améliorer les performances de décodage des turbo codes. Néanmoins, ces approches sont coûteuses
à mettre en œuvre. C'est pourquoi les implémentations matérielles de telles solutions sont rares. Les travaux conduits 
durant cette thèse répondent alors à la problématique de proposer de nouveaux algorithmes de décodage des
turbo codes. Ceux-ci doivent permettre une amélioration des performances de décodage tout en limitant le surcoût de l'implémentation matérielle 
des solutions de décodage mises en œuvre.
 La réponse à cette problématique qui fait le travail de thèse est récapitulé dans ce manuscrit. Ce dernier est organisé 
 en quatre chapitres.

\section*{Structure du manuscrit de thèse}
Le premier chapitre expose les concepts et les notions associées aux travaux de thèse. Dans un premier temps, les notions 
essentielles des codes correcteurs d'erreurs sont abordées. Celles-ci sont illustrées par le théorème du codage de 
canal, une présentation des différentes familles de codes correcteurs d'erreurs et enfin des métriques permettant la 
caractérisation des performances de décodage d'un code correcteur d'erreurs. Dans un second temps, les turbo codes sont détaillés.
La construction des turbo codes est tout d'abord exposée. Dans un second temps, le principe de turbo décodage est expliqué. Ceci amène 
l'introduction de la problématique du plancher d'erreurs. Les raisons de son 
apparition ainsi que les propositions de la littérature quant à sa réduction sont alors détaillées.

Le deuxième chapitre est consacré à l'étude des oscillations de métriques impliquées dans le décodage itératif des 
turbo codes. Suite à une observation fine de celles-ci, un algorithme les exploitant est proposé. Celui-ci, inspiré 
d'une approche originellement développée pour les codes LDPC, permet d'améliorer les performances de décodage de turbo codes dans la 
zone de convergence. Cependant, ces améliorations ne sont pas observées pour toutes les familles de turbo codes. 
Finalement, une tentative d'utilisation des oscillations est considérée dans un contexte de décodage répété d'une même 
trame.

Le troisième chapitre traite de la contribution majeure de ces travaux de thèse. Elle consiste en la présentation d'une méthode 
permettant de corriger les erreurs résiduelles rencontrées dans la zone du plancher d'erreurs lors du décodage de turbo 
codes. À la faveur d'une prédiction analytique, les erreurs résiduelles sont caractérisées. Ceci permet alors la 
comparaison de différentes métriques permettant de les détecter. Une de ces métriques, grâce à son fort pouvoir 
d'identification, est alors retenue comme cœur de la proposition d'un algorithme de correction des erreurs résiduelles. 
Cet algorithme a pour propriété d'abaisser drastiquement le plancher d'erreurs de tous les turbo codes standardisés
qui ont pu être considérés dans ce manuscrit. 

Le quatrième chapitre est dédié à la description d'une architecture matérielle adaptée à l'algorithme de correction 
des erreurs résiduelles. Cet algorithme, par son principe, peut être vu comme une extension d'un turbo décodeur. Dès 
lors les différents ordonnancements de turbo décodage sont successivement présentés. Suite à cela, afin d'assurer la maîtrise de la complexité 
calculatoire de cet algorithme, une étude des différents paramètres 
sur les performances de décodages est mené. Celle-ci permet alors la proposition d'une architecture 
matérielle de référence adaptée à un ordonnancement particulier de turbo décodage. Le coût matériel de l'architecture 
est comparé à celui d'une architecture matérielle de turbo décodage équivalente. Finalement, des projections sont 
esquissées quant aux modifications requises à l'adaptation de cette architecture aux autres ordonnancements de turbo 
décodage existants.

\section*{Contributions des travaux de thèse}
Les différentes contributions originales de ces travaux de thèse sont : 
\begin{enumerate}
	\item La proposition d'un algorithme de décodage permettant l'augmentation de la convergence de décodage de 
	turbo codes dans certains contextes. Son principe est d'annuler la contribution des décodeurs élémentaires du 
	turbo décodeur lorsqu'un changement de signe sur cette contribution est détectée d'une itération à l'autre. Dans le
	cadre du standard CCSDS, des gains de $0,1$ dB sont observés en comparaison d'un turbo décodage conventionnel.
	%Ceci est détaillé dans le deuxièmne chapitre.
	\item La proposition d'un algorithme de décodage basé sur le décodage successif d'une même trame permettant une 
	réduction limitée du plancher d'erreurs. Un abaissement d'un ordre de grandeur pour de hautes valeurs du rapport signal à bruit est atteint pour différents turbo codes. Cependant les gains sont obtenus au prix d'un surcoût calculatoire important. 
	\item La formalisation d'une prédiction de la distribution des erreurs résiduelles dans la zone du plancher 
	d'erreurs des turbo codes. Celle-ci se base directement sur le spectre de distances des turbo codes. Le spectre est
	plus 
	facile à obtenir que les fonctions recenseuses de poids jusqu'alors utilisées.
	\item La proposition d'un algorithme de décodage permettant une réduction drastique du plancher d'erreurs des 
	turbo codes standardisés grâce à la correction des erreurs résiduelles. Cet algorithme, nommé Flip and Check a un surcoût en 
	terme de complexité calculatoire maîtrisé. Il permet d'abaisser le plancher d'erreurs d'au moins un ordre de grandeur 
	dans tous les contextes applicatifs considérés dans ce manuscrit.
	\item La proposition d'une architecture matérielle adaptée à l'algorithme Flip and Check de correction des erreurs résiduelles
	.
	Cette architecture matérielle démontre la possible intégration de la technique proposée pour des systèmes contraints, 
	grâce à un impact limité au niveau de la complexité calculatoire. De plus, la latence du 
	processus de turbo décodage n'est pas impacté.\\
\end{enumerate}
Ces différentes contributions ont fait l'objet de publications scientifiques : \\
\begin{itemize}
	\item Communication sans acte :
	\begin{itemize}
		\item T. Tonnellier, C. Leroux, B. L. Gal, C. Jego, B. Gadat, and C. Poulliat, “Extension
		du principe self-corrected de l’information extrinsèque au décodage itératif de turbo
		codes,” in GRETSI, Sept 2015.
		\item T. Tonnellier, C. Leroux, B. L. Gal, C. Jego, B. Gadat, and N. V. Wambeke,“Correction 
		des erreurs résiduelles lors du processus de turbo décodage : algorithme et architecture,”
		in GDR Soc-SiP, June 2017.\\
	\end{itemize}
	\item Communications internationales avec actes :
	\begin{itemize}
		\item T. Tonnellier, C. Leroux, B. L. Gal, C. Jego, B. Gadat, and N. V. Wambeke,
		“Lowering the error floor of double-binary turbo codes : The flip and check algorithm,”
		in 2016 9th International Symposium on Turbo Codes and Iterative Information
		Processing (ISTC), Sept 2016, pp. 156–160.
		\item A. Cassagne, T. Tonnellier, C. Leroux, B. L. Gal, O. Aumage, and D. Barthou,
		“Beyond gbps turbo decoder on multi-core cpus,” in 2016 9th International Sympo-
		sium on Turbo Codes and Iterative Information Processing (ISTC), Sept 2016, pp. 136–140.
		\item T. Tonnellier, C. Leroux, B. L. Gal, C. Jego, B. Gadat, and N. V. Wambeke,
		“Hardware architecture for lowering the error floor of lte turbo codes,” in 2016
		Conference on Design and Architectures for Signal and Image Processing (DASIP),
		Oct 2016, pp. 107–112.\\
	\end{itemize}
	\item Communication dans une revue internationale avec comité de lecture :
	\begin{itemize}
		\item T. Tonnellier, C. Leroux, B. L. Gal, B. Gadat, C. Jego, and N. V. Wambeke,
		“Lowering the error floor of turbo codes with crc verification,” IEEE Wireless
		Communications Letters, vol. 5, no. 4, pp. 404–407, Aug 2016.
	\end{itemize}
\end{itemize}


% Les codes correcteurs d'erreurs sont une des solutions permettant d'améliorer la qualité des communications 
% numériques. Leur principe est d'introduire de la redondance dans la séquence d'information binaire afin de corriger 
% les erreurs de transmission durant la réception de l’information. Les turbo codes sont une des familles de codes 
% correcteurs d'erreurs parmi les plus performantes. Ils sont donc employés dans différents standards de communications
% numériques adressant des contextes applicatifs divers et variés. L'une des métriques de performance d'un code 
% correcteur d'erreurs est sa capacité de correction en fonction de la qualité de transmission. La région du plancher 
% d'erreur, inhérente à tout turbo décodage, est particulièrement limitante lors d'une nécessité de très faibles taux 
% d'erreurs.

% Lorsqu'un turbo code est choisi pour un standard de communications numérique, ses performances sont adaptées aux
% contraintes applicatives correspondant au cas d'emploi de ce standard. Cependant, les besoins applicatifs peuvent 
% évoluer au cours du temps et diverger de ceux originellement considérés. Ainsi, des exigences de plus faibles taux 
% d'erreur à qualité de transmission égale peuvent être nécessaires. Deux solutions sont alors envisageables. La première
% consiste à repenser le code correcteur d'erreur afin d'atteindre ces nouveaux besoins. Néanmoins, cette approche est 
% particulièrement coûteuse. De fait, l'ensemble de l’infrastructure est rendue obsolète et un nouveau déploiement 
% d'équipements, que ce soit pour la transmission ou la réception, devient nécessaire. Cette solution est par exemple 
% difficilement envisageable pour un contexte de communications satellitaires.

% Une autre solution consiste à modifier uniquement la partie réception. Dans ce cas, les modifications des performances 
% se situent au niveau du turbo décodeur. Ces dernières années, la communauté scientifique a proposé différentes 
% approches permettant d'améliorer les performances de décodage des turbo codes. Néanmoins, ces approches sont coûteuses
% à mettre en œuvre. De fait, les implémentations matérielles de telles solutions sont rares. Les travaux conduits 
% durant cette thèse répondent alors à la problématique de proposer de nouveaux algorithmes de décodage des
% turbo codes permettant une amélioration de leurs performances dont leur implémentation matérielle peut être envisagée 
% sans limitation majeure. Pour répondre à cette problématique, le manuscrit est organisé en quatre chapitre.

% \section*{Structure du manuscrit}
% Le premier chapitre expose les concepts utilisés pour ces travaux de thèse. Dans un premier temps, les notions 
% essentielles des codes correcteurs d'erreurs sont abordées. Celles-ci sont constituées par le théorème du codage de 
% canal, une présentation des différentes familles de codes correcteurs d'erreurs et enfin des métriques permettant la 
% caractérisation des performances d'un code correcteur d'erreurs. Dans un second temps, les turbo codes sont détaillés.
% À partir d'un point de vue historique, la construction des turbo codes est exposée. Ensuite, le décodage des turbo 
% codes est détaillé. Ceci amène alors l'introduction de la problématique du plancher d'erreur. Les raisons de son 
% apparition ainsi que les propositions de la littérature quant à sa réduction sont alors décrites.

% Le deuxième chapitre est consacré à l'étude des oscillations de métriques impliquées dans le décodage itératif des 
% turbo codes. Suite à une observation fine de celles-ci, un algorithme les exploitant est proposé. Celui-ci, inspiré 
% d'une approche originellement développée pour les codes LDPC, permet d'améliorer les performances de décodage de turbo codes dans la 
% zone de convergence. Cependant, ces améliorations ne sont pas rencontrées pour toutes les familles de turbo codes. 
% Finalement, une tentative d'utilisation des oscillations est considérée dans un contexte de décodage répété d'une même 
% trame.

% Le troisième chapitre constitue la contribution majeure de ces travaux. Elle consiste en la présentation d'une méthode 
% permettant de corriger les erreurs résiduelles rencontrées dans le plancher d'erreurs lors du décodage des turbo 
% codes. À la faveur d'une prédiction analytique, les erreurs résiduelles sont caractérisées. Ceci permet alors la 
% comparaison de différentes métriques permettant de les détecter. Une de ces métriques, grâce à son fort pouvoir 
% d'identification, est alors au cœur de la proposition d'un algorithme de correction des erreurs résiduelles. 
% Cet algorithme a pour propriété d'abaisser drastiquement le plancher d'erreurs de tous les turbo codes standardisés
% considérés dans ce manuscrit. 

% Le quatrième chapitre est dédié à la description d'une architecture matérielle adaptée à l'algorithme de correction 
% des erreurs résiduelles. Cet algorithme, par son principe, peut être vu comme une extension d'un turbo décodeur. Dès 
% lors les différents ordonnancements de turbo décodage sont présentés. Suite à cela, l'étude des différents paramètres 
% de l'algorithme sur les performances de décodages est étudié. Ceci permet alors la proposition d'une architecture 
% matérielle de référence adaptée à un ordonnancement particulier de turbo décodage. Le coût matériel de l'architecture 
% est alors comparé à celle d'une architecture matérielle de turbo décodage basée sur cet ordonnancement. Finalement, des projections sont 
% esquissées quant aux modifications requises à l'adaptation de cette architecture à d'autres ordonnancements de turbo 
% décodage.

% \section*{Contributions}
% Les contributions originales de ces travaux de thèse sont les suivants : 
% \begin{enumerate}
% 	\item La proposition d'un algorithme de décodage permettant l'augmentation de la convergence de décodage de 
% 	turbo codes dans certains contextes.
% 	\item La proposition d'un algorithme de décodage basé sur le décodage successif d'une même trame permettant une 
% 	réduction limitée du plancher d'erreurs.
% 	\item La formalisation d'une prédiction de la distribution des erreurs résiduelles dans la zone du plancher 
% 	d'erreurs des turbo codes.
% 	\item La proposition d'un algorithme de décodage permettant une réduction drastique du plancher d'erreurs des 
% 	turbo codes standardisés grâce à la correction des erreurs résiduelles.
% 	\item La proposition d'une architecture matérielle adaptée à l'algorithme de correction des erreurs résiduelles.
% \end{enumerate}
% Ces différentes contributions ont fait l'objet de publications scientifiques listées en fin de ce manuscrit (cf. \hyperref[sec:mespublis]{Liste des Publications}).


% Claude Berrou : \og Pour Shannon, "amour" et "haine" ne sont que deux mots de cinq lettres pris dans un alphabet qui en compte 26. 
% Et les réponses aux questions "Dieu existe-t-il ?" ou "Va-t-il pleuvoir sur Brest aujourd'hui ?" apportent l'une comme l'autre 
% un bit d'information \fg.