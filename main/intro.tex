%!TEX root = ../my_thesis.tex
\chapter*{Introduction}
\markboth{Introduction}{Introduction}
\addcontentsline{toc}{chapter}{Introduction}

\section*{Contexte et problématique}
\section*{Structure du manuscrit de thèse}
\section*{Contributions des travaux de thèse}
Les différentes contributions originales de ces travaux de thèse sont : 
\begin{enumerate}
	\item 
\end{enumerate}
Ces différentes contributions ont fait l'objet de publications scientifiques : \\
\begin{itemize}
	\item Communications nationales avec actes :
	\begin{itemize}
     	\item 
	\end{itemize}
	\item Communications internationales avec actes :
	\begin{itemize}
		\item 
	\end{itemize}
	\item Communication dans une revue internationale avec comité de lecture :
	\begin{itemize}
		\item 
	\end{itemize}
\end{itemize}


% Les codes correcteurs d'erreurs sont une des solutions permettant d'améliorer la qualité des communications 
% numériques. Leur principe est d'introduire de la redondance dans la séquence d'information binaire afin de corriger 
% les erreurs de transmission durant la réception de l’information. Les turbo codes sont une des familles de codes 
% correcteurs d'erreurs parmi les plus performantes. Ils sont donc employés dans différents standards de communications
% numériques adressant des contextes applicatifs divers et variés. L'une des métriques de performance d'un code 
% correcteur d'erreurs est sa capacité de correction en fonction de la qualité de transmission. La région du plancher 
% d'erreur, inhérente à tout turbo décodage, est particulièrement limitante lors d'une nécessité de très faibles taux 
% d'erreurs.

% Lorsqu'un turbo code est choisi pour un standard de communications numérique, ses performances sont adaptées aux
% contraintes applicatives correspondant au cas d'emploi de ce standard. Cependant, les besoins applicatifs peuvent 
% évoluer au cours du temps et diverger de ceux originellement considérés. Ainsi, des exigences de plus faibles taux 
% d'erreur à qualité de transmission égale peuvent être nécessaires. Deux solutions sont alors envisageables. La première
% consiste à repenser le code correcteur d'erreur afin d'atteindre ces nouveaux besoins. Néanmoins, cette approche est 
% particulièrement coûteuse. De fait, l'ensemble de l’infrastructure est rendue obsolète et un nouveau déploiement 
% d'équipements, que ce soit pour la transmission ou la réception, devient nécessaire. Cette solution est par exemple 
% difficilement envisageable pour un contexte de communications satellitaires.

% Une autre solution consiste à modifier uniquement la partie réception. Dans ce cas, les modifications des performances 
% se situent au niveau du turbo décodeur. Ces dernières années, la communauté scientifique a proposé différentes 
% approches permettant d'améliorer les performances de décodage des turbo codes. Néanmoins, ces approches sont coûteuses
% à mettre en œuvre. De fait, les implémentations matérielles de telles solutions sont rares. Les travaux conduits 
% durant cette thèse répondent alors à la problématique de proposer de nouveaux algorithmes de décodage des
% turbo codes permettant une amélioration de leurs performances dont leur implémentation matérielle peut être envisagée 
% sans limitation majeure. Pour répondre à cette problématique, le manuscrit est organisé en quatre chapitre.

% \section*{Structure du manuscrit}
% Le premier chapitre expose les concepts utilisés pour ces travaux de thèse. Dans un premier temps, les notions 
% essentielles des codes correcteurs d'erreurs sont abordées. Celles-ci sont constituées par le théorème du codage de 
% canal, une présentation des différentes familles de codes correcteurs d'erreurs et enfin des métriques permettant la 
% caractérisation des performances d'un code correcteur d'erreurs. Dans un second temps, les turbo codes sont détaillés.
% À partir d'un point de vue historique, la construction des turbo codes est exposée. Ensuite, le décodage des turbo 
% codes est détaillé. Ceci amène alors l'introduction de la problématique du plancher d'erreur. Les raisons de son 
% apparition ainsi que les propositions de la littérature quant à sa réduction sont alors décrites.

% Le deuxième chapitre est consacré à l'étude des oscillations de métriques impliquées dans le décodage itératif des 
% turbo codes. Suite à une observation fine de celles-ci, un algorithme les exploitant est proposé. Celui-ci, inspiré 
% d'une approche originellement développée pour les codes LDPC, permet d'améliorer les performances de décodage de turbo codes dans la 
% zone de convergence. Cependant, ces améliorations ne sont pas rencontrées pour toutes les familles de turbo codes. 
% Finalement, une tentative d'utilisation des oscillations est considérée dans un contexte de décodage répété d'une même 
% trame.

% Le troisième chapitre constitue la contribution majeure de ces travaux. Elle consiste en la présentation d'une méthode 
% permettant de corriger les erreurs résiduelles rencontrées dans le plancher d'erreurs lors du décodage des turbo 
% codes. À la faveur d'une prédiction analytique, les erreurs résiduelles sont caractérisées. Ceci permet alors la 
% comparaison de différentes métriques permettant de les détecter. Une de ces métriques, grâce à son fort pouvoir 
% d'identification, est alors au cœur de la proposition d'un algorithme de correction des erreurs résiduelles. 
% Cet algorithme a pour propriété d'abaisser drastiquement le plancher d'erreurs de tous les turbo codes standardisés
% considérés dans ce manuscrit. 

% Le quatrième chapitre est dédié à la description d'une architecture matérielle adaptée à l'algorithme de correction 
% des erreurs résiduelles. Cet algorithme, par son principe, peut être vu comme une extension d'un turbo décodeur. Dès 
% lors les différents ordonnancements de turbo décodage sont présentés. Suite à cela, l'étude des différents paramètres 
% de l'algorithme sur les performances de décodages est étudié. Ceci permet alors la proposition d'une architecture 
% matérielle de référence adaptée à un ordonnancement particulier de turbo décodage. Le coût matériel de l'architecture 
% est alors comparé à celle d'une architecture matérielle de turbo décodage basée sur cet ordonnancement. Finalement, des projections sont 
% esquissées quant aux modifications requises à l'adaptation de cette architecture à d'autres ordonnancements de turbo 
% décodage.

% \section*{Contributions}
% Les contributions originales de ces travaux de thèse sont les suivants : 
% \begin{enumerate}
% 	\item La proposition d'un algorithme de décodage permettant l'augmentation de la convergence de décodage de 
% 	turbo codes dans certains contextes.
% 	\item La proposition d'un algorithme de décodage basé sur le décodage successif d'une même trame permettant une 
% 	réduction limitée du plancher d'erreurs.
% 	\item La formalisation d'une prédiction de la distribution des erreurs résiduelles dans la zone du plancher 
% 	d'erreurs des turbo codes.
% 	\item La proposition d'un algorithme de décodage permettant une réduction drastique du plancher d'erreurs des 
% 	turbo codes standardisés grâce à la correction des erreurs résiduelles.
% 	\item La proposition d'une architecture matérielle adaptée à l'algorithme de correction des erreurs résiduelles.
% \end{enumerate}
% Ces différentes contributions ont fait l'objet de publications scientifiques listées en fin de ce manuscrit (cf. \hyperref[sec:mespublis]{Liste des Publications}).


% Claude Berrou : \og Pour Shannon, "amour" et "haine" ne sont que deux mots de cinq lettres pris dans un alphabet qui en compte 26. 
% Et les réponses aux questions "Dieu existe-t-il ?" ou "Va-t-il pleuvoir sur Brest aujourd'hui ?" apportent l'une comme l'autre 
% un bit d'information \fg.