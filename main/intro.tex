%!TEX root = ../my_thesis.tex
\chapter*{Introduction}
\markboth{Introduction}{Introduction}
\addcontentsline{toc}{chapter}{Introduction}

Le décodage de codes polaires sur des architectures dites programmables constitue le sujet de cette thèse. Deux axes principaux sont développés. Le premier est l'implémentation d'algorithmes de décodage sur des architectures de processeurs généralistes. Le second est la conception d'architectures programmables spécialisées pour une classe d'applications, à savoir le décodage de codes polaires. Dans cette introduction, le contexte des codes correcteurs d'erreurs et de leurs architectures est établi puis le sujet de recherche est présenté et motivé. La structure du manuscrit est ensuite détaillée en récapitulant les différentes contributions contenues dans ces travaux de thèse.

\section*{Les codes correcteurs d'erreurs}

Une chaîne de communications numériques élémentaire est constituée d'une source, d'un canal de communication et d'un destinataire. Un message est émis par la source à travers le canal jusqu'au destinataire. \`A moins d'un canal idéal, des perturbations détériorent le message et des erreurs apparaissent. Ces erreurs peuvent être corrigées à l'aide de codes correcteurs d'erreurs. Pour ce faire, de la redondance est ajoutée au message d'origine, en aval de la source. Cette redondance est utilisée, en amont du destinataire, afin de corriger les erreurs potentielles.

Claude Shannon propose en 1948 une formalisation mathématique de l'information~\cite{shannon_mathematical_2001}. Un des résultats majeurs de ses travaux est l’existence d'une limite théorique à la quantité d'information transmissible sur un canal de communication. Cette limite peut être atteinte à l'aide d'un code correcteur d'erreurs dont la construction reste à déterminer.
Dès lors, la communauté scientifique s'empare de cette problématique.
\`A la même époque, John Bardeen, Walter Brattain et William Shockley inventent le transistor. Les circuits électroniques constituent alors l'outil nécessaire à la mise en œuvre de codes correcteurs d'erreurs. Ainsi, de 1948 à nos jours, les chercheurs et ingénieurs mettent au point des algorithmes et des architectures de systèmes de correction d'erreurs de plus en plus complexes, permettant de s'approcher de la limite théorique établie par Claude Shannon.

Les codes polaires sont une des familles de codes correcteurs d'erreurs les plus récentes. Ils sont inventés en 2008 par Erdal Ar{\i}kan~\cite{arikan_channel_2009}. Leur sélection pour faire partie de la prochaine norme 5G atteste de leur efficacité et montre la nécessité de concevoir des systèmes performants pour en réaliser les deux étapes : le codage et le décodage. Jusqu'à présent, dans une grande majorité des systèmes de communications mobiles, le décodage des codes correcteurs d'erreurs est réalisé à l'aide d'architectures matérielles dédiées. Ces architectures sont très efficaces en matière de débit, de latence et de consommation énergétique. Elles souffrent cependant d'un manque de flexibilité.


La description logicielle des algorithmes de décodage et leur exécution sur des processeurs généralistes apparaît comme une alternative prometteuse palliant ce manque de flexibilité. Des travaux récents montrent que de telles implémentations permettent d'atteindre de hauts débits associés à de faibles latences~\cite{sarkis_fast_2014,giard_fast_2014}. Cette flexibilité se traduit par une meilleure évolutivité des systèmes et une intégration facilitée. Il est ainsi plus aisé de distribuer le calcul, pour s'approcher d'une infrastructure du type Cloud. Les travaux décrits dans cette thèse ont pour but d'étudier et de proposer des solutions pour le décodage de codes polaires sur des architectures programmables.

\section*{Structure du manuscrit de thèse}

Le premier chapitre a pour thème les codes polaires. Après une brève introduction portant sur la composition d'une chaîne de communications numériques, les codes polaires et le principe de polarisation sont présentés. Ensuite, les différents algorithmes de décodage de codes polaires sont détaillés. Enfin, les méthodes d'élagage de l'arbre, qui constituent des versions simplifiées des algorithmes de décodage, sont présentées.

Un décodeur logiciel générique et flexible de codes polaires est proposé dans le chapitre~\ref{chap:soft_scl}. Il implémente les algorithmes de décodage à liste. Les concepts de généricité et de flexibilité du décodeur sont définis. La \textit{généricité} du décodeur est sa capacité à supporter une très grande variété de codes polaires, en matière de taille de mot de code, de construction et de concaténation avec des codes détecteurs d'erreurs. Un décodeur générique peut donc s'adapter à un très grand nombre de schémas de codage. La \textit{flexibilité} d'un décodeur est définie comme sa capacité à configurer l'algorithme de décodage dans le but de proposer des compromis entre débits, latences et performances de décodage. La version de l'algorithme, la taille de liste, le format de représentation des données internes sont des exemples de telles configurations. Plusieurs améliorations algorithmiques et d'implémentation sont appliquées à ce décodeur \textit{générique} et \textit{flexible}. Ainsi, il approche, atteint voire dépasse les décodeurs logiciels de l'état de l'art, selon les algorithmes et les niveaux de bruit du canal considéré. Ce décodeur est intégré au sein de la suite logicielle libre AFF3CT~\citemine{cassagne_gdr_2017}, le code source est donc disponible. Ceci permet son utilisation par l'ensemble de la communauté, ainsi que la reproduction de l'intégralité des résultats présentés dans ce deuxième chapitre.

Un défaut des décodeurs logiciels de ce type est toutefois l'énergie consommée. En effet, les processeurs considérés sont généralistes. Cela implique une grande complexité matérielle et une forte consommation énergétique. La seconde partie de nos travaux a pour but de proposer deux architectures programmables à faible consommation et haute performance pour le décodage de codes polaires. Le but de ces travaux est de conserver des architectures flexibles, tout en améliorant l'efficacité énergétique. Les chapitres ~\ref{chap:tensilica} et~\ref{chap:tta} reposent sur deux méthodologies de conception différentes. Par conséquent, la structure des architectures résultantes ainsi que leurs performances sont radicalement différentes.

Dans le chapitre~\ref{chap:tensilica}, la méthodologie considérée est celle de la spécialisation d'une architecture programmable de base. Il s'agit donc d'une approche par extension. Cette spécialisation est effectuée à l'aide des outils de la société Tensilica. Elle consiste à configurer une architecture de processeur de base puis à étendre son jeu d'instructions par l'ajout d'unités matérielles dédiées. Les processeurs ainsi conçus atteignent des débits comparables à ceux obtenus sur les processeurs d'architecture ARM, tout en réduisant la consommation énergétique d'un ordre de grandeur. 

Dans le chapitre~\ref{chap:tta}, la méthode de conception retenue permet une configuration plus fine du processeur. Cette approche implique la description de l'ensemble de l'architecture programmable. Sa structure ainsi que les modèles matériels des unités élémentaires du processeur sont alors entièrement définis par l'utilisateur. L'architecture programmable obtenue permet d'atteindre de très hauts débits et de faibles latences tout en réduisant fortement la complexité matérielle et donc la puissance dissipée. L'architecture proposée fait partie de la famille des TTA (Transport Triggered Architecture). Les résultats d'implémentations montrent que les décodeurs proposés dépassent les débits des implémentations logicielles sur architecture x86 et réduisent la consommation énergétique de deux ordres de grandeur tout en conservant une généricité et une flexibilité suffisante dans le contexte applicatif considéré.

\section*{Contributions des travaux de thèse}
Les différentes contributions originales de ces travaux de thèse peuvent se résumer comme suit : 
\begin{enumerate}
	\item La proposition d'un décodeur logiciel pour les algorithmes de décodage de codes polaires à liste. Les caractéristiques principales de ce décodeur logiciel sont sa généricité et sa flexibilité. Les débits obtenus dépassent les débits des implémentations logicielles de la littérature sur des architectures de processeurs x86. Des résultats d'exécution sur architecture ARM sont également présentés, constituant la première référence de la littérature. Ces travaux sont détaillés dans le deuxième chapitre.
  	\item La proposition de la première architecture ASIP (Application Specific Instruction-set Processor) spécialisée dans le décodage de codes polaires. Cette architecture permet de conserver une grande flexibilité tout en améliorant l'efficacité énergétique. Les débits obtenus sont comparables à ceux de l'architecture ARM. La consommation énergétique est réduite d'un ordre de grandeur. Ces travaux sont présentés dans le troisième chapitre.
  	\item La proposition de la première architecture du type TTA (Transport Triggered Architecture) pour le décodage de codes polaires. Il s'agit d'un type spécifique d'ASIP. Tout comme les architectures précédentes, elle bénéficie d'une grande flexibilité. Elle permet une exploitation importante du parallélisme d'instructions et une définition fine de la structure. Ainsi, les débits obtenus dépassent ceux obtenus sur les architectures de processeur x86 et la consommation énergétique est quant à elle réduite de deux ordres de grandeur. Ces travaux sont détaillés dans le quatrième chapitre.

\end{enumerate}

\vspace{1cm}
Ces différentes contributions ont été valorisées à travers des publications scientifiques : \\

\begin{itemize}
		\item[$\bullet$] Conférences nationales :
	\begin{itemize}
		\item[$\bullet$] M. Léonardon, C. Leroux, and C. Jégo, “Les Codes polaires, Algorithmes et Décodeurs.”
		CNES CCT TSI Technologies pour la 5G - segment spatial, 2016.
     	\item[$\bullet$] A. Cassagne, M. Léonardon, O. Hartmann, T. Tonnellier, G. Delbergue, V. Giraud, C. Leroux, R. Tajan, B. Le Gal, C. Jégo, O. Aumage, and D. Barthou, “AFF3CT : Un Environnement de Simulation pour le Codage de Canal,” GdR SoC2, 2017.
	\end{itemize}
	\item[$\bullet$] Conférences internationales :
	\begin{itemize}
		\item[$\bullet$]  A. Cassagne, M. Léonardon, O. Hartmann, G. Delbergue, T. Tonnellier, R. Tajan, C. Leroux, C. Jégo, B. Le Gal, O. Aumage, and D. Barthou, “Fast Simulation and Prototyping With AFF3CT,” in IEEE International Workshop on Signal Processing Systems (SiPS), 2017.
		\item[$\bullet$] A. Ghaffari, M. Léonardon, Y. Savaria, C. Jégo, and C. Leroux, “Improving Performance of SCMA MPA Decoders Using Estimation of Conditional Probabilities,” in IEEE International New Circuits and Systems Conference (NEWCAS), 2017.
		\item[$\bullet$] M. Léonardon, C. Leroux, D. Binet, J. M. P. Langlois, C. Jégo, and Y. Savaria, “Custom Low Power Processor for Polar Decoding,” in IEEE International Symposium on Circuits and Systems (ISCAS), 2018.
		\item[$\bullet$] M. Léonardon, C. Leroux, P. Jääskeläinen, C. Jégo, and Y. Savaria, “Transport Triggered Polar Decoders,” in IEEE International Symposium on Turbo Codes and Iterative Information Processing (ISTC), 2018.
	\end{itemize}
	\item[$\bullet$] Revues internationales avec comité de lecture :
	\begin{itemize}
		\item[$\bullet$] M. Léonardon, A. Cassagne, C. Leroux, C. Jégo, L.-P. Hamelin, and Y. Savaria, “Fast and Flexible Software Polar List Decoders,” Journal of Signal Processing Systems (JSPS), 2018, version 2 révisée en cours de revue (août 2018).
        \item[$\bullet$] A. Ghaffari, M. Léonardon, A. Cassagne, C. Leroux, and Y. Savaria, “Toward High Performance Implementation of 5G SCMA Algorithms,” soumis à IEEE Access, 2018.
	\end{itemize}
\end{itemize}

