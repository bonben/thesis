\documentclass{standalone}
\usepackage[T1]{fontenc}
\usepackage[utf8]{inputenc}
\usepackage[french]{babel}
\usepackage{eulervm}
\usepackage{pgfplots}
\pgfplotsset{compat=newest}
\usepgfplotslibrary{colorbrewer}
\usepgfplotslibrary{groupplots}
\usetikzlibrary{matrix, positioning}
\usepackage{pgffor}
\usepackage{amsmath}

\setlength{\textwidth}{146.8mm}


\begin{document}

\begin{tikzpicture}
		
		\begin{groupplot}[group style={group name=chev_lte, group size= 2 by 3, horizontal sep=1cm, vertical sep=2cm}, 
					height=0.4\textwidth,  width=0.62\textwidth,
					xmin=0, xmax=30, xtick={0,5,...,30},
					ymin=0, ymax=500,
					grid=both, grid style={gray!30},
					colorbrewer cycle list=Spectral,
					xlabel=Itérations
					]
		\nextgroupplot[ylabel=Oscillations $N\rightarrow N$]
			\foreach \n in {0,...,99}{\addplot+[mark=none]  table [x=Ite, y=f\n] {../chevelures/lte/10-5/nn_f.dat}; }	
		\nextgroupplot[xmax=25, xtick={0,5,...,25}, width=0.5\textwidth]
			\foreach \n in {0,...,99}{\addplot+[mark=none]  table [x=Ite, y=f\n] {../chevelures/lte/10-5/nn_c.dat}; }	
		\nextgroupplot[ylabel=Oscillations $N\rightarrow I$]
			\foreach \n in {0,...,99}{\addplot+[mark=none]  table [x=Ite, y=f\n] {../chevelures/lte/10-5/ni_f.dat}; }	
		\nextgroupplot[xmax=25, xtick={0,5,...,25}, width=0.5\textwidth]
			\foreach \n in {0,...,99}{\addplot+[mark=none]  table [x=Ite, y=f\n] {../chevelures/lte/10-5/ni_c.dat}; }
		\nextgroupplot[ylabel=Nombre d'erreurs binaires]
			\foreach \n in {0,...,99}{\addplot+[mark=none]  table [x=Ite, y=f\n] {../chevelures/lte/10-5/err_f.dat}; }	
		\nextgroupplot[xmax=25, xtick={0,5,...,25}, width=0.5\textwidth]
			\foreach \n in {0,...,99}{\addplot+[mark=none]  table [x=Ite, y=f\n] {../chevelures/lte/10-5/err_c.dat}; }		
\end{groupplot}
		\node[below = 1cm of chev_lte c1r1.south] {(a) : Trames erronées};
        \node[below = 1cm of chev_lte c2r1.south] {(b) : Trames corrigées};
		\node[below = 1cm of chev_lte c1r2.south] {(c) : Trames erronées};
        \node[below = 1cm of chev_lte c2r2.south] {(d) : Trames corrigées};
        \node[below = 1cm of chev_lte c1r3.south] {(e) : Trames erronées};
        \node[below = 1cm of chev_lte c2r3.south] {(f) : Trames corrigées};
\end{tikzpicture}

\end{document}
