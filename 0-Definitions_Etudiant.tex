%!TEX root = ./Document.tex
%% -----------------------------------
%% ---> À MODIFIER PAR L'ETUDIANT / TO BE MODIFIED BY THE STUDENT <---
%% -----------------------------------
%%
%% Commandes qui affichent le titre du document, le nom de l'auteur, etc.
\newcommand\monTitre{Décodage de codes polaires sur des architectures programmables.}
\newcommand\monPrenom{Mathieu}
\newcommand\monNom{Léonardon}
\newcommand\monDepartement{Génie \'Electrique}  % Department
\newcommand\maDiscipline{\'Electronique}
\newcommand\monDiplome{D}        % (M)aîtrise ou (D)octorat / (M)aster or Ph(D)
\newcommand\anneeDepot{2018}    % Year
\newcommand\moisDepot{Novembre}       % Month
\newcommand\monSexe{M}           % "M" ou "F" = Gender
\newcommand\PageGarde{N}         % "O" ou "N" = Yes or No
\newcommand\AnnexesPresentes{O}  % "O" ou "N". Indique si le document comprend des annexes. / If the thesis includes annexes = O or N = No.
\newcommand\mesMotsClef{Codes polaires, Décodeur Logiciel, ASIP, Annulation Successive, Annulation Successive à Liste}
%%
%%  DEFINITION DU JURY
%%
%%  Pour la définition du jury, les macros suivantes sont definies:
%%  \PresidentJury, \DirecteurRecherche, \CoDirecteurRecherche, \MembreJury, \MembreExterneJury
%%
%%  Toutes les macros prennent 4 paramètres: Sexe (M/F), Prénom, Nom, Titres
\newcommand\monJury{\PresidentJury{M}{Mohamad}{Sawan}{Professeur Titulaire}\\
\DirecteurRecherche{M}{Yvon}{Savaria}{Professeur Titulaire}\\
\DirecteurRecherche{M}{Christophe}{J\'ego}{Professeur Titulaire}\\
\MembreJury{M}{Amer}{Baghdadi}{Professeur}\\
\MembreJury{M}{Emmanuel}{Casseau}{Professeur des Universités}\\
\MembreJury{M}{Camille}{Leroux}{Ma\^itre de Conf\'erences}\\
\MembreJury{M}{Olivier}{Muller}{Ma\^itre de Conf\'erences}\\
\MembreJury{M}{Charly}{Poulliat}{Professeur des Universités}}

\newcommand\templateVersion{EPM}
