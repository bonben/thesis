\chapter*{Liste des notations}
\markboth{Liste des notations}{Liste des notations}
\addcontentsline{toc}{chapter}{Liste des notations}
% put your text here
\begin{center}
%\begin{tabular}{lL{10cm}}    
\begin{longtable}{ p{.20\textwidth}  p{.80\textwidth} } 
    $B_{i,j}$               & Fonction retournant: $\dfrac{i}{2^{j}} \text{~mod}~ 2$.   \\
    $K$                     & Nombre de bits d'information.                                                                                                                                                         \\
    $F$                     & Fonction LR: $F(L_1,L_2)=\dfrac{1+L_1L_2}{L_1+L_2}$ .                                                                                                                                 \\
    $f$                     & Fonction LLR: $f(\lambda_1,\lambda_2)=2\tanh^{-1}(\tanh(\dfrac{\lambda_a}{2})\tanh(\dfrac{\lambda_b}{2})) \approx sgn(\lambda_a\times\lambda_b)\times min(|\lambda_a|,|\lambda_b|)$ . \\
    $G$                     & Fonction LR: $G(L_1,L_2,S_{i,j})=L_1*L_2^{1-2S_{i,j}}$ .                                                                                                                              \\
    $g$                     & Fonction LLR: $g(\lambda_1,\lambda_2,S_{i,j})=\lambda_b+(-1)^{S}\lambda_a$ .                                                                                                          \\
    $H_l$                   & Fonction qui récupère les sommes partielles du noeud inférieur.                                                                                                                       \\
    $H_u$                   & Fonction qui récupère les sommes partielles du noeud supérieur.                                                                                                                       \\
    $L_{i,j}$               & LR en $i^{\text{ème}}$ ligne et $j^{\text{ème}}$ colonne dans le \textit{factor graph}.                                                                                               \\
    $N$                     & Taille du code.                                                                                                                                                                       \\
    $n$                     & $\log_2(N)$.                                                                                                                                                                          \\
    $P$                     & Niveau de parallélisme. Nombre d'éléments de calcul dans l'unité de traitement.                                                                                                       \\
    $Q=(Q_{c},Q_{i},Q_{f})$ & Quantification des LLRs. Représentation en signe-magnitude.                                                                                                                           \\
    $Q_{c}$                 & Nombre de bit de la partie entière des LLRs issus du canal.                                                                                                                           \\
    $Q_{i}$                 & Nombre de bit de la partie entière des LLRs internes.                                                                                                                                 \\
    $Q_{f}$                 & Nombre de bit de la partie fractionnaire, commun aux LLRs issus du canal et aux LLRs internes.                                                                                        \\
    $R$                     & $R=\dfrac{K}{N}$, rendement du code.                                                                                                                                                  \\
    $S_{i,j}$                & Somme partielle en $i^{\text{ème}}$ ligne et $j^{\text{ème}}$ colonne dans le \textit{factor graph}.                                                                                  \\
    $s$                     & Indice de l'étage de séparation entre le cluster combinatoire et le cluster binaire dans l'architecture mixte.                                                                        \\
    $T$                     & Nombre d'arbres combinatoires de l'architecture mixte.                                                                                                                                \\
    $U$                     & Message à transmettre contenant $K$ bits d'information et $N-K$ bits gelés.                                                                                                           \\
    $X$                     & Version codée du message U.                                                                                                                                                           \\
	$Y$                     & Version bruitée de X à la sortie du canal de transmission.\\   
		 $B_{i,j}$               & Fonction retournant: $\dfrac{i}{2^{j}} \text{~mod}~ 2$.   \\
    $K$                     & Nombre de bits d'information.                                                                                                                                                         \\
    $F$                     & Fonction LR: $F(L_1,L_2)=\dfrac{1+L_1L_2}{L_1+L_2}$ .                                                                                                                                 \\
    $f$                     & Fonction LLR: $f(\lambda_1,\lambda_2)=2\tanh^{-1}(\tanh(\dfrac{\lambda_a}{2})\tanh(\dfrac{\lambda_b}{2})) \approx sgn(\lambda_a\times\lambda_b)\times min(|\lambda_a|,|\lambda_b|)$ . \\
    $G$                     & Fonction LR: $G(L_1,L_2,S_{i,j})=L_1*L_2^{1-2S_{i,j}}$ .                                                                                                                              \\
    $g$                     & Fonction LLR: $g(\lambda_1,\lambda_2,S_{i,j})=\lambda_b+(-1)^{S}\lambda_a$ .                                                                                                          \\
    $H_l$                   & Fonction qui récupère les sommes partielles du noeud inférieur.                                                                                                                       \\
    $H_u$                   & Fonction qui récupère les sommes partielles du noeud supérieur.                                                                                                                       \\
    $L_{i,j}$               & LR en $i^{\text{ème}}$ ligne et $j^{\text{ème}}$ colonne dans le \textit{factor graph}.                                                                                               \\
    $N$                     & Taille du code.                                                                                                                                                                       \\
    $n$                     & $\log_2(N)$.                                                                                                                                                                          \\
    $P$                     & Niveau de parallélisme. Nombre d'éléments de calcul dans l'unité de traitement.                                                                                                       \\
    $Q=(Q_{c},Q_{i},Q_{f})$ & Quantification des LLRs. Représentation en signe-magnitude.                                                                                                                           \\
    $Q_{c}$                 & Nombre de bit de la partie entière des LLRs issus du canal.                                                                                                                           \\
    $Q_{i}$                 & Nombre de bit de la partie entière des LLRs internes.                                                                                                                                 \\
    $Q_{f}$                 & Nombre de bit de la partie fractionnaire, commun aux LLRs issus du canal et aux LLRs internes.                                                                                        \\
    $R$                     & $R=\dfrac{K}{N}$, rendement du code.                                                                                                                                                  \\
    $S_{i,j}$                & Somme partielle en $i^{\text{ème}}$ ligne et $j^{\text{ème}}$ colonne dans le \textit{factor graph}.                                                                                  \\
    $s$                     & Indice de l'étage de séparation entre le cluster combinatoire et le cluster binaire dans l'architecture mixte.                                                                        \\
    $T$                     & Nombre d'arbres combinatoires de l'architecture mixte.                                                                                                                                \\
    $U$                     & Message à transmettre contenant $K$ bits d'information et $N-K$ bits gelés.                                                                                                           \\
    $X$                     & Version codée du message U.                                                                                                                                                           \\
	$Y$                     & Version bruitée de X à la sortie du canal de transmission.                                                                                                                                  
%\end{tabular}
\end{longtable}

\end{center}
\vspace*{\fill}
