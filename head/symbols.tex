%!TEX root = ../my_thesis.tex
\chapter*{Liste des notations}
\markboth{Liste des notations}{Liste des notations}
\addcontentsline{toc}{chapter}{Liste des notations}
% put your text here
\begin{center}

\begin{longtable}{ p{.18\textwidth}  p{.82\textwidth} } 

$E_b$         	 		&	Énergie moyenne par bit d'information     												\\
$\mathbf{c}$			& 	Mot de code 																			\\
$C$						& 	Capacité du canal de transmission														\\
$\mathbf{\hat{d}}$		&	Mot décidé																				\\
$\text{erfc}(\cdot)$    &	Fonction d'erreur complémentaire     													\\
$K$     	  			&	Taille du mot d'information non codé     												\\
$L(\cdot)$				&	Valeur LLR 																				\\
$N$     	 			& 	Taille du mot de code                													\\
$R$     	 			&	Rendement du code ($R=K/N$)       														\\
$\sigma$				&   Variance du bruit																		\\
$\text{sgn}(\cdot)$    	&	Fonction signe																			\\


		
\end{longtable}

\end{center}
\chapter*{Liste des acronymes}
\markboth{Liste des acronymes}{Liste des acronymes}
\addcontentsline{toc}{chapter}{Liste des acronymes}
% put your text here
\begin{center}

\begin{longtable}{ p{.18\textwidth}  p{.82\textwidth} } 

AWGN 		&   Additive White Gaussian Noise	 																	\\
BE 			& 	Bit Error, nombre d'erreurs binaires	 															\\
BER 		&   Bit Error Rate																						\\
BPSK 		&	Binary Phase-Shift Keying																			\\
CRC 		& 	Cyclic Redundancy Check 																			\\
FE 			& 	Frame Error	 																						\\
FER 		&   Frame Error Rate																					\\
LDPC 		& 	Low Density Parity Check Codes	 																	\\
LLR 		&  	Log Likelihood Ratio	 																			\\
LL 	 		&  	Log Likelihood	 																					\\
LTE			& 	Long Term Evolution 																				\\
RAM			& 	Random Access Memory																				\\
SNR			&	Signal-to-Noise Ratio																				\\


\end{longtable}

\end{center}
%\vspace*{\fill}

