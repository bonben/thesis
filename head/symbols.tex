%!TEX root = ../my_thesis.tex
\chapter*{Liste des notations}
\markboth{Liste des notations}{Liste des notations}
\addcontentsline{toc}{chapter}{Liste des notations}
% put your text here
\begin{center}

\begin{longtable}{ p{.18\textwidth}  p{.82\textwidth} } 

$\alpha$       			&	Métrique d'état (ou de nœud) aller 				  										\\
$\argmax(\cdot)$       	&	Nombre de mots de code de poids de Hamming $d$  										\\
$A_d$       	 		&	Nombre de mots de code de poids de Hamming $d$  										\\
$\beta$       			&	Métrique d'état (ou de nœud) retour				  										\\
$B$       	        	&   Degré de parallélisme de sous-bloc    													\\
$d_{min}$    	 		&	Distance minimale du code        														\\
$E_b$         	 		&	Énergie moyenne par bit d'information     												\\
$\mathbf{c}$			& 	Mot de code 																			\\
$C$						& 	Capacité du canal de transmission														\\
$\mathbf{\hat{d}}$		&	Mot décidé																				\\
$\text{erfc}(\cdot)$    &	Fonction d'erreur complémentaire     													\\
$\gamma$       			&	Métrique de branche				 				  										\\
$H(X)$					& 	Entropie de la variable aléatoire $X$													\\
$H(X,Y)$				& 	Entropie conjointe des variables aléatoires $X$ et $Y$									\\
$H(X|Y)$				& 	Entropie conditionnelle de la variable aléatoire $X$ sachant $Y$						\\
$K$     	  			&	Taille du mot d'information non codé     												\\
$L(\cdot)$				&	Valeur LLR 																				\\
$\maxstar(\cdot,\cdot)$ & 	Opérateur max-étoile																	\\
$\nu$					& 	Nombre de mémoires d'un code convolutif													\\
$N	\rightarrow N$ 		&	Oscillation dans le domaine naturel entre deux itérations								\\
$N	\rightarrow I$ 		&	Oscillation entre le domaine naturel et le domaine entrelacé							\\
$N_0$    	 			&	Densité spectrale mono-latérale du bruit    											\\
$N$     	 			& 	Taille du mot d'information codé      													\\
$\Pi$					& 	Fonction d'entrelacement																\\
$q$      	        	&   Taille du vecteur des positions les moins fiables pour l'algorithme FNC 				\\  																			
$R$     	 			&	Rendement du code ($R=K/N$)       														\\
$\sigma$				&   Variance du bruit																		\\
$\text{sgn}(\cdot)$    	&	Fonction signe																			\\
$W_d$       	 		&	Somme des poids de Hamming des $A_d$ séquences de poids de Hamming $d$ 					\\
$W$       	       		&	Nombre de fenêtres pour un SISO avec fenêtre glissante        							\\


		
\end{longtable}

\end{center}
\chapter*{Liste des acronymes}
\markboth{Liste des acronymes}{Liste des acronymes}
\addcontentsline{toc}{chapter}{Liste des acronymes}
% put your text here
\begin{center}

\begin{longtable}{ p{.18\textwidth}  p{.82\textwidth} } 

APP 		& 	A Posteriory Probability																			\\
ARP 		& 	Almost Regular Permutation 																			\\
ARQ 		& 	Automatic Repeat-reQuest	 																		\\
AWGN 		&   Additive White Gaussian Noise	 																	\\
BCJR 		&	Algorithme de décodage MAP nommé d'après ses inventeurs : Bahl, Cocke, Jelinek et Raviv				\\
BE 			& 	Bit Error, nombre d'erreurs binaires	 																\\
BER 		&   Bit Error Rate																						\\
BF 			&	Backward-Forward																					\\
BFLY 		&	Butterfly																							\\
BPSK 		&	Binary Phase-Shift Keying																			\\
CCSDS 		&   Consultative Committee for Space Data Systems														\\
CIM 		&  	Correction Impulse Method																			\\
CRC 		& 	Cyclic Redundancy Check 																			\\
DRP 		&   Dithered Relatively Prime																			\\
DVB-RCS 	&   Digital Video Broadcasting - Return Channel via Satellite	 										\\
EML-MAP 	&   Enhanced max log - Maximum A Posteriori																\\
FB 			&	Forward-Backward																					\\
FDM			&   Fonction De Masse																					\\
FE 			& 	Frame Error	 																						\\
FER 		&   Frame Error Rate																					\\
FNC 		&   Flip and Check																						\\
FSM 		&   Forced Symbol Method																				\\
LDPC 		& 	Low Density Parity Check Codes	 																	\\
LIFO 		& 	Last-In First-Out																					\\
LLR 		&  	Log Likelihood Ratio	 																			\\
LL 	 		&  	Log Likelihood	 																			\\
LTE			& 	Long Term Evolution 																				\\
LVA			&	List Viterbi Algorithm	 																			\\
MAP			&   Maximum A Posteriori 																				\\
ML			&   Maximum Likelihood : maximum à vraisemblance 														\\
OSC			& 	Oscillations																						\\
OSD			& 	Ordered Statistic Decoding 																			\\
QPP			& 	Quadratic Permutation Polynomial	 																\\
QPSK		&	Quadrature Phase-Shift Keying																		\\
RAM			& 	Random Access Memory																				\\
RSC			& 	Recursive Systemactic Convolutionnal																\\
SB			&	Sub-Block																							\\
SC			&   Self-Corrected																						\\
SISO		&   Soft Input Soft Output																				\\
SNR			&	Signal-to-Noise Ratio																				\\
SOVA		&	Soft Output Viterbi Algorithm																		\\
SW			&	Sliding-Window																						\\
WEF			&   Weight Enumerating Function 																		\\


\end{longtable}

\end{center}
%\vspace*{\fill}

