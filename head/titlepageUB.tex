\documentclass[a4paper, 11pt]{article}
\usepackage[utf8]{inputenc}
\usepackage[T1]{fontenc}
\usepackage[french]{babel}
\usepackage[top=4.5cm,bottom=1.5cm,right=1.5cm,left=1.5cm]{geometry}
\usepackage{tikz}
\usepackage{eulervm}
\definecolor{bleuUni}{RGB}{0, 157, 224}
\definecolor{marronUni}{RGB}{68, 58, 49}
\usetikzlibrary{calc}
\begin{document}
  \thispagestyle{empty}
 % \setlength{\oddsidemargin}{-2mm}
   \begin{center}
  %   %\large
    
  %   %\sffamily
    
    
    %\tikzset{external/export next=false}
    \begin{tikzpicture}[overlay,remember picture]
      \coordinate (SW) at (current page.south west);
      \coordinate (SE) at (current page.south east);
      \coordinate (NW) at (current page.north west);
      \coordinate (NE) at (current page.north east);
      
      \coordinate (anchorSceauA) at ($(SW)!.84!(SE)$);
      \coordinate (anchorSceauB) at ($(SE)!0.25!(NE)$);
      \coordinate (anchorSceau) at (anchorSceauA |- anchorSceauB);
      
      \node[opacity=.1] at (anchorSceau) {\includegraphics[height=7.5cm]{sceauUnivBord.png}};
      \node[inner sep=5,anchor=north west] at ($(NW)!.05!(SE)$) {\includegraphics[height=2.5cm]{ub.jpg}};
      
      \node[inner sep=5,anchor=north east] at ($(NE)!.05!(SW)$) {\includegraphics[height=2.5cm]{TAS.jpg}};
    \end{tikzpicture}
    
    \noindent \Large{\\THÈSE PRÉSENTÉE\\POUR OBTENIR LE GRADE DE\\}
      \vspace*{1.5em}
      \noindent \Huge \textbf{DOCTEUR DE \\L'UNIVERSITÉ DE BORDEAUX\\} 
        \vspace*{1.5em}
        \noindent \Large{\bsc{École Doctorale des Sciences de l'Ingénieur\\Spécialité : Électronique\\}}
          \vspace*{1.5em}
          \noindent \Large{par \textbf{Thibaud \bsc{Tonnellier}}\\}
            \vspace*{1.5em}
            {\color{bleuUni}\hrule} \vspace*{0.2cm}
            % \noindent {\Huge \textbf{Mise en \oe uvre d'un environnement de \\ prototypage temps réel pour l'expertise \\ \vspace*{.3ex} de nouveaux schémas de codage canal}} \\
            \noindent {\fontsize{24}{24} \textbf{Contribution à l'amélioration des \\\vspace*{.3ex}performances de décodage des turbo codes : \\\vspace*{.3ex}algorithmes et architecture}}
            \vspace*{0.2cm} {\color{bleuUni}\hrule}
            \vspace*{1.5em}
            \begin{tabular}{rl}
              \Large{Directeur de thèse : }     & Christophe \bsc{Jégo} \\
              \Large{Co-encadrants de thèse : } & Bertrand \bsc{Le Gal}  \\
                                                 & Camille \bsc{Leroux}   \\
            \end{tabular}
            
            \vspace*{1.5em}
            \noindent \Large préparée au Laboratoire IMS \\ en collaboration avec Thales Alenia Space \\
            \vspace*{1.5em}
            \noindent \large soutenue le \\
            \vspace*{1.5em}
          \end{center}
          \noindent \large \textbf{Jury :} \\
          \vspace*{-1.5em}
          \begin{center}
            \begin{tabular}{lclclr}
              Jean-Pierre \textsc{Cances}  &-& Professeur des Universités & - & ENSIL                & \textit{Rapporteur}   \\
              Michel \textsc{Jézéquel}     & - & Professeur de l'IMT & - & IMT Atlantique                & \textit{Rapporteur}   \\
              Charly \textsc{Poulliat}          & - & Professeur des Universités                      & - & INP Toulouse        & \textit{Examinateur}   \\
              Nicolas \textsc{Van Wambeke} & - & Docteur Ingénieur                  & - & Thales Alenia Space & \textit{Examinateur}  \\
              %Benjamin \textsc{Gadat}      & - & Ingénieur                  & - & Airbus Group        & \textit{Examinateur}  \\
              Bertrand \textsc{Le Gal}     & - & Maître de Conférence      & - & Bordeaux INP        & \textit{Co-directeur} \\
              Camille \textsc{Leroux}      & - & Maître de Conférence      & - & Bordeaux INP        & \textit{Co-directeur} \\
              Christophe \textsc{Jégo}     & - & Professeur des Universités & - & Bordeaux INP        & \textit{Directeur}    \\
            \end{tabular}
          \end{center}
          
          
          
        \end{document}
        
