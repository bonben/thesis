%!TEX root = ../my_thesis.tex
%\cleardoublepage
\addcontentsline{toc}{chapter}{Résumés}
\chapter*{Résumé}
%\markboth{Résumé}{Résumé}
% put your text here
\vskip1em


Les codes polaires constituent une classe de codes correcteurs d'erreurs inventés récemment qui suscite l'intérêt des chercheurs et des industriels, comme en atteste leur sélection pour le codage des canaux de contrôle dans la prochaine génération de téléphonie mobile (5G). Un des enjeux des futurs réseaux mobiles est la virtualisation des traitements numériques du signal, et en particulier les algorithmes de codage et de décodage.  Afin d'améliorer la flexibilité du réseau, ces algorithmes doivent être décrits de manière logicielle et être déployés sur des architectures programmables. Une telle infrastructure  de réseau permet de mieux répartir l'effort de calcul sur l'ensemble des \noeuds et d'améliorer la coopération entre cellules. Ces techniques ont pour but de réduire la consommation d'énergie, d'augmenter le débit et de diminuer la latence des communications. Les travaux présentés dans ce manuscrit portent sur l'implémentation logicielle des algorithmes de décodage de codes polaires et la conception d'architectures programmables spécialisées pour leur exécution.

Une des caractéristiques principales d'une chaîne de communication mobile est l'instabilité du canal de communication. Afin de remédier à cette instabilité, des techniques de modulations et de codages adaptatifs sont utilisées dans les normes de communication. Ces techniques impliquent que les décodeurs supportent une vaste gamme de codes : ils doivent être génériques. La première contribution de ces travaux est l'implémentation logicielle de décodeurs génériques des algorithmes de décodage "à Liste" sur des processeurs à usage général. En plus d'être génériques, les décodeurs proposés sont également flexibles. Ils permettent en effet des compromis entre pouvoir de correction, débit et latence de décodage par la paramétrisation fine des algorithmes. En outre, les débits des décodeurs proposés atteignent les performances de l'état de l'art et, dans certains cas, les dépassent.

La deuxième contribution de ces travaux est la proposition d'une nouvelle architecture programmable performante spécialisée dans le décodage de codes polaires. Elle fait partie de la famille des processeurs à jeu d'instructions dédiés à l'application. Un processeur de type RISC à faible consommation en constitue la base. Cette base est ensuite configurée, son jeu d'instructions est étendu et des unités matérielles dédiées lui sont ajoutées. Les simulations montrent que cette architecture atteint des débits et des latences proches des implémentations logicielles de l'état de l'art sur des processeurs à usage général. La consommation énergétique est réduite d'un ordre de grandeur. En effet, lorsque l'on considère le décodage par annulation successive d'un code polaire (1024,512), l'énergie nécessaire par bit décodé est de l'ordre de 10 nJ sur des processeurs à usage général contre 1 nJ sur les processeurs proposés.

La troisième contribution de ces travaux est également une architecture de processeur à jeu d'instructions dédié à l'application. Elle se différencie de la précédente par l'utilisation d'une méthodologie de conception alternative. Au lieu d'être basée sur une architecture de type RISC, l'architecture du processeur proposé fait partie de la classe des architectures déclenchées par le transport. Elle est caractérisée par une plus grande modularité qui permet d'améliorer très significativement l'efficacité du processeur. Les débits mesurés sont alors supérieurs à ceux obtenus sur les processeurs à usage général. La consommation énergétique est réduite à environ 0.1 nJ par bit décodé pour un code polaire (1024,512) avec l'algorithme de décodage par annulation successive. Cela correspond à une réduction de deux ordres de grandeur en comparaison de la consommation mesurée sur des processeurs à usage général.
 
\vskip0.5cm
\emph{Mots clefs :} Codes polaires, Décodeur Logiciel, ASIP, Annulation Successive, Annulation Successive à Liste
\cleardoublepage
\chapter*{Abstract}
\vskip1em

Polar codes are a recently invented class of error-correcting codes that are of interest to both researchers and industry, as evidenced by their selection for the coding of control channels in the next generation of cellular mobile communications (5G). One of the challenges of future mobile networks is the virtualization of digital signal processing, including channel encoding and decoding algorithms. In order to improve network flexibility, these algorithms must be written in software and deployed on programmable architectures. Such a network infrastructure allow dynamic balancing of the computational effort across the network, as well as inter-cell cooperation. These techniques are designed to reduce energy consumption, increase throughput and reduce communication latency. The work presented in this manuscript focuses on the software implementation of polar codes decoding algorithms and the design of programmable architectures specialized in their execution.

One of the main characteristics of a mobile communication chain is that the state of communication channel changes over time. In order to address issue, adaptive modulation and coding techniques are used in communication standards. These techniques require the decoders to support a wide range of codes: they must be generic. The first contribution of this work is the software implementation of generic decoders for "List" polar decoding algorithms on general purpose processors. In addition to their genericity, the proposed decoders are also flexible. Trade-offs between correction power, throughput and decoding latency are enabled by fine-tuning the algorithms. In addition, the throughputs of the proposed decoders achieve state-of-the-art performance and, in some cases, exceed it.

The second contribution of this work is the proposal of a new high-performance programmable architecture specialized in polar code decoding. It is part of the family of Application Specific Instruction-set Processors (ASIP). The base architecture is a RISC processor. This base architecture is then configured, its instruction set is extended and dedicated hardware units are added. Simulations show that this architecture achieves throughputs and latencies close to state-of-the-art software implementations on general purpose processors. Energy consumption is reduced by an order of magnitude. The energy required per decoded bit is about 10 nJ on general purpose processors compared to 1 nJ on proposed processors when considering the Successive Cancellation (SC) decoding algorithm of a polar code (1024,512).

The third contribution of this work is also the design of an ASIP architecture. It differs from the previous one by the use of an alternative design methodology. Instead of being based on a RISC architecture, the proposed processor architecture is part of the class of Transport Triggered Architectures (TTA). It is characterized by a greater modularity that allows to significantly improve the efficiency of the processor. The measured flow rates are then higher than those obtained on general purpose processors. The energy consumption is reduced to about 0.1 nJ per decoded bit for a polar code (1024,512) with the SC decoding algorithm. This corresponds to a reduction of two orders of magnitude compared to the consumption measured on general purpose processors.

\vskip0.5cm
\emph{Key words:} Polar-codes...
