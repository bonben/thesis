%!TEX root = ../my_thesis.tex
%\cleardoublepage
\addcontentsline{toc}{chapter}{Résumés}
\chapter*{Résumé}
%\markboth{Résumé}{Résumé}
% put your text here
\vskip1em


Les codes correcteurs d'erreurs sont des algorithmes indispensable dans tout système de communication ou de stockage numérique.
Les codes polaires constituent une classe de codes correcteurs d'erreurs qui suscitent l'intérêt des chercheurs et des industriels, comme en atteste leur sélection pour le codage des canaux de contrôle dans la prochaine génération de téléphonie mobile (5G). 
Un des enjeux du nouveau réseau radio est la virtualisation des traitements numériques du signal, dont les algorithmes de codage et de décodage.  Afin d'améliorer la flexibilité du réseau, ces algorithmes de traitement doivent être décrits de manière logicielle et être déployés sur des architectures programmables. Une telle infrastructure  de réseau doit permettre l'équilibrage de l'effort de calcul sur l'ensemble des \noeuds ainsi que la coopération entre cellules. Ces techniques réduiront la consommation d'énergie, augmenteront le débit et de diminueront la latence des communications. Les travaux présentés dans ce manuscrit portent sur l'implémentation logicielle des algorithmes de décodage de codes polaires et la conception d'architecture programmables spécialisées pour leur exécution. 

La première contribution majeure est l'implémentation logicielle d'une famille d'algorithmes de décodage de codes polaires, dite "à Liste", sur des processeurs à usage général. 
Une des caractéristiques principales d'une chaîne de communication mobile est l'instabilité du canal de communication.
Afin de remédier à cette instabilité, des techniques de modulation et de codage adaptatifs sont utilisées dans les standards de communication.
Ces techniques nécessitent que les décodeurs supportent une vaste gamme de codes : ils doivent être génériques.
Les décodeurs proposés sont donc génériques mais également flexibles, car ils permettent des compromis entre pouvoir de correction, débit et latence de décodage par la paramétrisation fine des algorithmes.
De plus, les performances de débit de certains des implémentations de certains de ces algorithmes sont les plus élevées à ce jour sur les cibles considérées.

La deuxième contribution majeure est la proposition d'une nouvelle architecture programmable performante spécialisée dans le décodage de codes polaires. Elle fait partie de la famille des processeurs à jeu d'instruction dédiés à l'application. Un processeur de type RISC à faible consommation en constitue la base. Cette base est ensuite configurée finement, son jeu d'instruction est étendu et des unités matérielles dédiées lui sont ajoutées. Les simulations montrent que cette architecture atteint les débits et les latences des implémentations logicielles de l'état de l'art. La consommation énergétique est réduite d'un ordre de grandeur.

La troisième contribution majeure est également une architecture de processeur à jeu d'instruction dédié à l'application. Elle se différencie de la précédente par l'utilisation d'une méthodologie de conception alternative. Au lieu d'être basée sur une architecture RISC, l'architecture du processeur proposé fait partie de la classe des architectures déclenchées par le transport. Elle est caractérisée par une plus grande modularité. Cette modularité permet d'améliorer l'efficacité du processeur. Les débits mesurés sont alors supérieurs à ceux obtenus sur les processeurs à usage général, et la consommation énergétique est réduite de deux ordres de grandeur.
 
\vskip0.5cm
\emph{Mots clefs :} Codes polaires, Décodeur Logiciel, ASIP, Annulation Successive, Annulation Successive à Liste
\cleardoublepage
\chapter*{Abstract}
\vskip1em

English abstract.

\vskip0.5cm
\emph{Key words:} Polar-codes...
